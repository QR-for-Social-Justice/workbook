%% LyX 2.3.6.2 created this file.  For more info, see http://www.lyx.org/.
%% Do not edit unless you really know what you are doing.
\documentclass[oneside,english]{amsart}
\usepackage[T1]{fontenc}
\usepackage{geometry}
\geometry{verbose,tmargin=1in,bmargin=1in,lmargin=1in,rmargin=1in}
\usepackage{amstext}
\usepackage{amsthm}

\makeatletter

%%%%%%%%%%%%%%%%%%%%%%%%%%%%%% LyX specific LaTeX commands.
%% Because html converters don't know tabularnewline
\providecommand{\tabularnewline}{\\}

%%%%%%%%%%%%%%%%%%%%%%%%%%%%%% Textclass specific LaTeX commands.
\numberwithin{equation}{section}
\numberwithin{figure}{section}
\theoremstyle{plain}
\newtheorem{thm}{\protect\theoremname}
\theoremstyle{definition}
\newtheorem{xca}[thm]{\protect\exercisename}
\theoremstyle{definition}
\newtheorem{defn}[thm]{\protect\definitionname}

\makeatother

\usepackage{babel}
\providecommand{\definitionname}{Definition}
\providecommand{\exercisename}{Exercise}
\providecommand{\theoremname}{Theorem}

\begin{document}
\title{WCSAM 206 11 - Breaking the Enigma}
\maketitle
\begin{xca}
\textbf{(reading question) }
\begin{enumerate}
\item Describe the role of the scramblers in the Enigma machine. What is
the benefit of adding multiple scramblers instead of just one? 
\begin{enumerate}
\item The scramblers encipher individual letters by connecting them via
wires. For instance, plaintext a might go to A in the scrambler, which
is connected by a wire to B, and the current goes out of the scrambler
into the B lamp.
\item Having multiple scramblers allows the cipher alphabet to change after
each letter is enciphered, since scramblers rotate each other after
encryption. Moreover, even after typing 26 letters, the cipher alphabet
won't go back to the beginning because the second rotor will rotate.
\end{enumerate}
\item What is the benefit of adding a reflector to the Enigma machine? 
\begin{enumerate}
\item The reflector sends the current back through the machine in a mirror-symmetric
way.
\item This means that to get plaintext back from ciphertext, all you have
to do is run the ciphertext through the machine.
\item Encryption and decryption are the same process!
\end{enumerate}
\item Name two additional Enigma features that made decrypting Enigma more
difficult. How did these features add to the security of the Enigma
machine?
\begin{enumerate}
\item The plugboard switched letters with each other to add to the number
of possible encryptions.
\item The scramblers are removable and interchangeable, increasing the number
of keys in a 3-rotor Enigma by a factor of $3!=6$.
\end{enumerate}
\end{enumerate}
\end{xca}


\section{One-time pads}
\begin{itemize}
\item The Enigma machine is a portable \textbf{one-time pad}: a Vigen\`{e}re
cipher with keyword a random string of letters as long as the message.
\item When implemented perfectly, without human error, one-time pads are
theoretically unbreakable. Do the following to determine why:
\end{itemize}
\begin{xca}
{[}slide{]} 
\begin{enumerate}
\item Use the one-time pad DNCR GZBQCSP QRXZGNP as the Vigen\`{e}re key
to decrypt the message JBQU SNSDKFV LZBSTNB . 
\item Now decrypt the same message using pad DNCR OSOQCSP LNXBLLB. 
\item Why are one-time pads theoretically unbreakable? 
\item Pick another plaintext message of the same length as the given ciphertext
and find a one-time pad so that decrypting the given ciphertext using
your one-time pad yields your plaintext.
\item What are some downsides to using a one-time pad on the battlefield? 
\end{enumerate}
\end{xca}


\section{{[}slides{]} Components of the Enigma Machine}
\begin{itemize}
\item {[}slides{]} demonstrate downloading \& installing the Enigma machine,
then run students through exercises
\item Enigma types:
\begin{itemize}
\item 3-rotor Heer (Army) and Luftwaffe (airforce) Wehrmacht Enigma I
\item Kriegsmarine (Navy) Enigma M3
\item Kriegsmarine (Navy) four-rotor Enigma M4
\end{itemize}
\end{itemize}
\begin{xca}
\textbf{(reading question) }Read p143-154 of \emph{The Code Book},
ending at the last full paragraph on p154 (the line \textquotedbl match
it to a suspect\textquotedbl ). Then answer the following questions,
explaining all reasoning, in a text box below.
\end{xca}

\begin{enumerate}
\item Why did Poland take a particular interest in breaking the German Enigma
codes in the period between WWI and WWII? 
\begin{enumerate}
\item It was sandwiched between conquering powers Germany and the USSR
\end{enumerate}
\item What is the difference between the day key and message key of an Enigma
machine? Why did the Germans not just use a single daily key? 
\begin{enumerate}
\item The day key is agreed upon beforehand and consists of three letters
(a scrambler orientation or \emph{grundstellung}) that are then used
to encipher and send the message key.
\item The message key is a random scrambler orientation chosen by the operator
and sent using the day key to the recipient. It can supposedly only
be decrypted by holders of the day key.
\item Using a single day key for all messages would mean a ton of day-key
ciphertext would be available to anyone who could intercept German
radio signals, providing ample fodder for cryptanalysts. Using both
a day and message key, in contrast, means that the day key encrypts
only $6$ letters per message sent on that day, providing very little
ciphertext for cryptanaylsis.
\end{enumerate}
\item What tactic did Marian Rejewski use to find the Enigma scrambler settings
on a given day?
\begin{enumerate}
\item The double repetition of the message key that began every message
gave two instances of each letter's ciphertext.
\item For example, if a message began LOKRGM, that means that L and R are
both encryptions of the first letter of the message key, hence L and
R are related by the day key.
\item By intercepting multiple messages and noting relationships between
letters, Rejewski constructed \textbf{chains} of ciphertext letters
(1st and 4th message key letters, for instance). 
\item For example, if A is the 1st letter in one message and F is the fourth,
then $A\to F$. Similarly, if F is the first letter in another message
and W is the fourth, then $A\to F\to W$. If $W\to A$, then we have
a chain with $3$ links: $A\to F\to W\to A$.
\item The number of links in the chain is purely a consequence of the scrambler
settings and has nothing to do with the plugboard settings. Swapping
plugboard settings changes some of the letters in each chain (e.g.
switching $A\to F\to W$ to $B\to C\to Q$ and vice versa), but not
the lengths of the chains.
\end{enumerate}
\end{enumerate}
\begin{xca}
{[}slides{]} play around with the Enigma simulator and explore the
components. 
\end{xca}


\section{Rotors (Walzen)}
\begin{itemize}
\item Each rotor interior encodes a monoalphabetic substitution cipher (swaps
a pair of plaintext letters with each other seemingly randomly)
\item {[}slide{]} By rotating the outer wheel, we're effectively changing
the encoded monoalphabetic cipher--$a$ now goes where $b$ used
to go under that rotor's cipher.
\item The complexity increases: there are $3$ rotor slots and $5$ different
rotors that can be swapped in and out for a Wehrmacht Enigma, and
$4$ slots for $8$ possible rotors in the navy (kriegsmarine).
\end{itemize}
\begin{xca}
How many possible ways are there of slotting the $5$ Wehrmacht rotors
into $3$ slots? $8$ Kriegsmarine rotors into $4$ slots?
\end{xca}

\begin{defn}
Each rotor has an \textbf{inner ring} which determines which letters
are wired to which, and an \textbf{outer ring} which connects each
letter of the inner ring with the previous/next Enigma component.

Changing the ringstellung rotates the positions of all the electrical
contacts, so that the wire that used to connect to A now connects
to B, changing the monoalphabetic substitution cipher implemented
by the rotor. Changing the grundstellung shifts the outer alphabet
ring, so that an A typed on the keyboard enters the rotor as a B.
Thus, the grundstellung applies an additional shift cipher offset
to the output of the rotor. 
\end{defn}

\begin{xca}
If each rotor has $26$ possible \emph{ringstellung} (inner ring positions
of the electrical contacts) and $26$ possible \emph{grundstellung}
(numbers facing up when we rotate the outer ring), how many possible
rotor settings are there on a Wehrmacht Enigma? Kriegsmarine?

How many possible ways are there to plug one cable into the plugboard
(connecting two letters)? Two cables? Ten? Be sure not to double-count
(e.g., $(ab)=(ba)$ and $(ab)(cd)=(cd)(ab)$).
\begin{itemize}
\item If order mattered, this wouldn't be too difficult--it's 
\[
26\times25\times24\times23\times22\times21\times20\times19\times18\times17=1.9275224\times10^{13},
\]
an enormous number of permutations!
\item However, order doesn't matter: plugging $(ab)$ is the same as plugging
$(ba)$. We'll have to \emph{divide} this huge number by the number
of duplicate plugboard settings. 
\item {[}slide{]} For one cable, there are $26\times25$ ordered ways of
plugging two letters together, but we have to divide by the duplicates
$AB=BA$, $2$ duplicates per setting: 
\[
\frac{26\times25}{2}=325\text{ possible settings}.
\]
\item For two cables, we can pick any four letters to put plugs into. Say
we choose $A,B,C,D$. Then the following plugboard settings are equivalent:
\[
\begin{array}{ccc}
(ab)(cd) & \leftrightarrow & (ab)(dc)\\
\updownarrow &  & \updownarrow\\
(ba)(cd) & \leftrightarrow & (ba)(dc)
\end{array}
\]
There are $2^{2}=4$ ways of transposing two pairs of two letters
each because each pair of letters adds a factor of $2$ to the number
of possible rearrangements.
\item But wait! We also have to account for the fact that $(ab)(cd)$ implies
we're plugging $A\leftrightarrow B$ and $C\leftrightarrow D$ simultaneously.
This is the same as $(cd)(ab)$, which is the same in turn as $(dc)(ab)$
etc..., so the whole picture is more like 
\[
\begin{array}{ccc}
(cd)(ab) & \leftrightarrow & (dc)(ab)\\
\updownarrow &  & \updownarrow\\
(cd)(ba) & \leftrightarrow & (dc)(ba)
\end{array}\leftrightarrow\begin{array}{ccc}
(ab)(cd) & \leftrightarrow & (ab)(dc)\\
\updownarrow &  & \updownarrow\\
(ba)(cd) & \leftrightarrow & (ba)(dc).
\end{array}
\]
\item There are $8$ duplicates per two-cable setting! This is because there
are $2=2!$ possible ways of placing two digrams (two-letter chunks)
into $2$ slots, and $2^{2}=4$ ways of transposing/arranging the
letters within each digram. So we have to multiply the number of possibilities
from transpositions ($4=2^{2}$ for $2$ letters, $2^{n}$ for $n$
letters) times the number of possible digram permutations $(n!$ for
$n$ letters):
\[
\frac{26\times25\times24\times23}{2^{2}\times2!}=44850\text{ settings}.
\]
\item In general, we have the following pattern:

\begin{tabular}{|c|c|c|c|}
\hline 
Number of cables & $2$ & $3$ & $10$\tabularnewline
\hline 
\hline 
Number of permutations & $\frac{26!}{(26-4)!}$ & $\frac{26!}{(26-6)!}$ & $\frac{26!}{(26-20)!}$\tabularnewline
\hline 
\# transpositions & $2^{2}$ & $2^{3}$ & $2^{10}$\tabularnewline
\hline 
\# rearrangements of pairs & $2!$ & $3!$ & $10!$\tabularnewline
\hline 
Number of settings & $\frac{26!}{(26-4)!\times2^{2}\times2!}=44850$ & $\frac{26!}{(26-6)!\times2^{3}\times3!}=3453450$ & $\frac{26!}{(26-20)!\times2^{10}\times10!}=150,738,275,000,000$\tabularnewline
\hline 
\end{tabular}
\item For $n$ cables, the formula for the number of possible plugboard
settings is:
\[
\frac{26!}{(26-2n)!\times2^{n}\times n!}=\frac{26\times25\times\dots\times(26-2n+1)}{2^{n}\times n!}
\]
\item If we use anywhere between $1$ and $10$ cables, we have to add all
of the previously-computed numbers of settings together:
\[
\sum_{n=1}^{10}\frac{26!}{(26-2n)!\times2^{n}\times n!}=2.1675106498\times10^{14}>200,000,000,000,000
\]
which is too many settings even for a modern computer to brute-force.
\item Compare to the $26\times26\times26=17,576$ possible ground positions
(grundstellung) and $26^{3}=17,576$ possible ring settings (ringstellung)
for three rotors. In total, three rotors have $26^{6}=308,915,776$
possible settings, while the plugboard has almost $1,000,000$ times
as many!
\end{itemize}
\end{xca}

\begin{itemize}
\item {[}slide{]} The Germans eventually got wise to this issue and only
stated the encrypted message key once in each message. However, the
Bletchley Park team headed by Turing was still able to break Enigma
messages.
\end{itemize}
\begin{xca}
Watch the Numberphile video about the flaw in the Enigma machine,
then answer the following questions in full sentences, explaining
your reasoning in each case.
\begin{enumerate}
\item Does the Enigma machine encrypt using a monoalphabetic or polyalphabetic
substitution cipher? Explain your answer. What mechanical aspect(s)
of the Enigma machine make it monoalphabetic or polyalphabetic? 
\begin{enumerate}
\item Pressing K repeatedly makes a different letter light up each time,
so the cipher is polyalphabetic.
\end{enumerate}
\item What was the major flaw in the Enigma machine that allowed its codes
to be broken? Why was this a flaw? 
\begin{enumerate}
\item No letter can be enciphered as itself.
\item This allows one to determine the possible positions of a crib.
\end{enumerate}
\item Describe in 3-4 sentences how the Bombe machine (called \textquotedbl Christopher\textquotedbl{}
in The Imitation Game) tests rotor positions in order to guess the
settings of the Enigma machine. 
\begin{enumerate}
\item The Bombe tries to work out the plugboard at the front of the machine.
\item The signal first goes through the plugboard, through the machine,
and back through the plugboard.
\end{enumerate}
\item What made naval Enigma codes harder to break than Army or Air Force
codes? 
\begin{enumerate}
\item The rotor starting positions (message key) were sent at the beginning
of each message, in another code (day key) entirely.
\end{enumerate}
\item If you had to explain to someone who knew no cryptography or probability
how the Enigma code was broken, what would you say?
\begin{enumerate}
\item Guess a word or phrase that might appear in an Enigma message, such
as WETTERBERICHT (weather report).
\item Write the word on a piece of paper, and slide your guess for the Enigma
code underneath until you find positions where it ``fits'', i.e.
has no letter enciphered as itself.
\item From this point, knowing say that $t\mapsto E$, work out the plugboard
settings by making a guess as to where a plug is (e.g. $T\leftrightarrow A$).
Deduce from this and the knowledge that
\[
t\to\text{Plugboard}\to^{a}\text{Rotors}\to^{P}\text{Plugboard}\to E
\]
that $(PE)$ is a connection. 
\item Do the same for the other letters in your crib. If you get a contradiction,
that two letters are connected to the same letter in the plugboard,
our initial guess must have been wrong. Repeat to eliminate all guesses
that give contradictions.
\item If all 26 options are wrong, that means the rotor position is wrong.
Change the rotor position and check the next position.
\item Turing's ideas helped save time:
\begin{enumerate}
\item Once you've found a contradiction, all guesses that led to the contradiction
are ``fruit of a poison tree''; you can discard them all.
\item The Bombe machine can very quickly eliminate incorrect deductions
with electrical circuits. What you're left with at the end is what
wasn't wrong.
\end{enumerate}
\end{enumerate}
\end{enumerate}
\end{xca}

\begin{itemize}
\item {[}slide{]} In a story that may be apocryphal (a legend), a German
officer stationed in North Africa would send periodic messages containing
the phrase \textquotedbl nothing to report\textquotedbl{} in German,
which translates to \emph{keine besonderen ereignisse}. 
\end{itemize}
\begin{xca}
Suppose that one of the Allied listening posts in North Africa intercepts
the following coded message: UAUNFYRLPZSWMEDSINFKRJXFSXKJCAXKEZ. You
suspect that this message contains the plaintext sequence KEINEBESONDERENEREIGNISSE. 
\begin{enumerate}
\item Give one example of a collection of 25 consecutive ciphertext letters
in the coded message that cannot decrypt to \textquotedbl KEINEBESONDERENEREIGNISSE\textquotedbl .
Explain your answer. Can you tell which ciphertext letters in the
coded message correspond to this plaintext? Explain your answer and
describe your reasoning.
\end{enumerate}
\end{xca}

\begin{itemize}
\item The German word ``EINS'', meaning ``one'', was found in 90\% of
plaintext Enigma messages.
\begin{itemize}
\item The Bletchley Park crew made a catalog of every possible Enigma encipherment
of ``EINS''. If one of those possible ciphertexts showed up in a
message, it was likely that it decrypted to ``EINS''.
\item From there, the crew could determine the Enigma settings.
\end{itemize}
\item By 1945, almost all German Enigma traffic could be decrypted within
a day or two.
\item The Germans were still confident of its security and openly discussed
their plans and movements.
\item After the war, it was learnt that German cryptographers were aware
that Enigma was not unbreakable, they just couldn't fathom that anyone
would go to such lengths to do it.
\end{itemize}
\begin{xca}
VigenereEncrypt Python module
\end{xca}


\end{document}
