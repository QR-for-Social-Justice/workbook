%% LyX 2.3.6.2 created this file.  For more info, see http://www.lyx.org/.
%% Do not edit unless you really know what you are doing.
\documentclass[oneside,english]{amsart}
\usepackage[T1]{fontenc}
\usepackage{geometry}
\geometry{verbose,tmargin=1in,bmargin=1in,lmargin=1in,rmargin=1in}
\usepackage{amstext}
\usepackage{amsthm}
\usepackage{graphicx}

\makeatletter
%%%%%%%%%%%%%%%%%%%%%%%%%%%%%% Textclass specific LaTeX commands.
\numberwithin{equation}{section}
\numberwithin{figure}{section}
\theoremstyle{plain}
\newtheorem{thm}{\protect\theoremname}
\theoremstyle{definition}
\newtheorem{xca}[thm]{\protect\exercisename}
\theoremstyle{definition}
\newtheorem{defn}[thm]{\protect\definitionname}
\theoremstyle{plain}
\newtheorem{lyxalgorithm}[thm]{\protect\algorithmname}
\theoremstyle{definition}
\newtheorem{example}[thm]{\protect\examplename}

\makeatother

\usepackage{babel}
\providecommand{\algorithmname}{Algorithm}
\providecommand{\definitionname}{Definition}
\providecommand{\examplename}{Example}
\providecommand{\exercisename}{Exercise}
\providecommand{\theoremname}{Theorem}

\begin{document}
\title{WCSAM 206 14 - RSA Encryption, or: Factoring Is Hard}
\maketitle

\section{Last Time}
\begin{itemize}
\item We saw that the Diffie-Hellman Key Exchange could be used to agree
on a single number, which could be used as a shared secret key.
\item However, there is no obvious way to \emph{undo} the process of the
key exchange, even if you're the intended recipient, and recover the
sender's private key.
\item In other words, there's no way to \emph{decrypt }a message under Diffie-Hellman,
since any plaintext taken to the power of two exponents $\mod p$
will be changed in a way that it's impossible to recover without knowing
the sender's secret key.
\item In $1973$, Clifford Cocks invented a scheme that allowed for an encrypted
message to be sent, and decrypted, \emph{without }any key exchange
that would be helpful to a cryptanalyst. It was equivalent to what
would be independently invented by Rivest, Shamir, and Adleman in
1977 and called RSA encryption after the first letters of their last
names.
\end{itemize}

\section{Reading question}
\begin{xca}
\textbf{(reading question)} Recall that a \textbf{trapdoor one-way
function} is a mathematical process that's easy to do forwards, hard
to undo, but easy to undo if you have a secret key (the \textquotedbl trapdoor\textquotedbl ).
As an example, consider the process of mixing colors described in
the Diffie-Hellman Color Exchange video.

Watch the delightfully cheesy (and 90s) video above, then answer the
following questions about what you've watched in a text box or document
below:
\end{xca}

\begin{enumerate}
\item In Clifford Cocks's cipher, what information is kept secret? 
\begin{enumerate}
\item the two prime numbers that are multiplied together in order to generate
the public modulus
\item hence, the inverse $e^{-1}\mod(p-1)(q-1)$
\end{enumerate}
\item In Clifford Cocks's cipher, what information is made public? 
\begin{enumerate}
\item the composite number $N=pq$, which is the product of two secret large
primes $p$ and $q$.
\item the base used for exponentiation.
\end{enumerate}
\item What one-way function does Cocks use in his algorithm to prevent anyone
who intercepts his messages from reading them? 
\begin{enumerate}
\item Really two:
\begin{enumerate}
\item multiplication of large prime numbers
\item modular exponentiation
\end{enumerate}
\end{enumerate}
\item Why is it hard to undo the one-way function in this case?
\begin{enumerate}
\item multiplication is easy, but factoring (especially factoring products
of very large primes) is hard
\item taking powers mod $N$ is easy (for computers); solving the discrete
log problem (without knowing $e^{-1}\mod(p-1)(q-1)$) is really hard!
\end{enumerate}
\end{enumerate}

\section{Overview}
\begin{itemize}
\item To do cryptography, we need a process (encryption) that's easy to
do forwards, but hard to do backwards (decryption) \textbf{unless
}you have a secret key. Such a process is called a \textbf{one-way
function}.
\item However, the process has to be easy to do backwards \emph{if you have
a secret key}, otherwise the intended recipient could never hope to
read the message. The use of such a key is called a \textbf{trapdoor},
and a one-way function with a trapdoor is understandably called a
\textbf{trapdoor one-way function.}
\item How did Cocks/RSA come up with their cipher? They first noted that
mathematicians had already come up with a way of solving the discrete
log problem in specific instances: if we happen to know that the modulus
$N$ is the product of two prime numbers $p,q$!
\item The primary one-way function in RSA is $m^{e}\mod pq$, where $m$
is the message, $e$ is a public key (published by whoever wants to
send messages using RSA), and $pq$ is the product of two very large
prime numbers.
\item This ultimately relies on \textbf{two }one-way functions.
\begin{enumerate}
\item It's easy (for computers) to take powers $\mod N$, where $N$ is
some large composite number, but it's hard to \textbf{undo }exponentiation
$\mod N$ (the discrete logarithm problem) because of the seemingly
random way in which modular exponentiation jumbles numbers.
\item It's easy (for computers) to multiply two very large prime numbers,
but hard to factor a large composite number that's the product of
two large primes.
\end{enumerate}
\item No non-quantum computer could ever hope to do either of these problems.
But then how would the intended recipient receive the message?
\item The trapdoor hinges entirely on Euler's Theorem:
\end{itemize}
\begin{defn}
\textbf{Euler's phi function }or \textbf{totient function }$\phi(n)$,
defined on a positive whole number $n$, is defined to be the number
of whole numbers less than or equal to $n$ that are relatively prime
to (share no factors other than $1$ with) $n$.
\begin{itemize}
\item We've already said that Euler's Thoerem states that, if $a$ is any
positive whole number and $p$ is a prime number that doesn't share
any factors with $a$, then $a^{p-1}\equiv1\mod p$. Note that $\phi(p)=p-1$,
since every positive whole number less than $p$ (a total of $p-1$
numbers) share no factors with $p$.
\item Euler's Theorem also says that, if $p$ and $q$ are prime numbers
and $a$ is any positive whole number so that $\gcd(a,pq)=1$, then
$a^{(p-1)(q-1)}\equiv1\mod pq$. It's true in general that $\phi(pq)=(p-1)(q-1)$
if $p,q$ are prime, since the $p-1$ numbers less than $p$, the
$q-1$ numbers less than $q$, and all the products thereof share
no factors with $pq$.
\item In fact, Euler's Theorem says more than that. It actually says the
following:
\end{itemize}
\end{defn}

\begin{thm}[Euler's Theorem]
 If $a$ and $n$ are positive whole numbers such that $\gcd(a,n)=1$,
then $a^{\phi(n)}\equiv1\mod n$.
\end{thm}

Euler's Theorem is actually a generalization of another theorem called
Fermat's Little Theorem.
\begin{thm}[Fermat's Little Theorem]
 If $n$ is a positive integer and $p$ is prime, then $n^{p}\equiv n\mod p$.
\end{thm}

\begin{proof}
By Euler's Theorem, $n^{p-1}\equiv1\mod p$. Hence, 
\[
n^{p}=(n^{p-1})n\equiv1n=n\mod p.
\]
\end{proof}
\begin{itemize}
\item Fermat's ``big theorem'' is his famous Last Theorem, which says
that the equation $a^{n}+b^{n}=c^{n}$ has no solutions for any integer
value of $n$ bigger than $2$. It was proved in $1995$ after centuries
of work, by mathematician Andrew Wiles, who won a million-dollar prize.
\item You may believe that it's hard to factor very large numbers, but how
well does modular exponentiation really jumble up numbers? Is it really
that hard to undo?
\end{itemize}
\includegraphics[scale=0.75]{\string"Old Cryptography Notes/pasted180\string".png}

\section{The Algorithm}
\begin{itemize}
\item Have people come up to the board to be Alice, Bob, and Eve.
\end{itemize}
\begin{lyxalgorithm}[RSA Encryption]
 (whisper the primes to Alice; announce the public info)
\begin{enumerate}
\item Alice wants to receive messages from Bob without Eve, who looks through
everyone's mail, intercepting it. Alice picks two huge prime numbers,
$p$ and $q$. In general, we want the primes to be enormous, but
for simplicity in this example we'll assume that Alice chooses $p=17,q=11$.
She keeps these numbers secret.
\item Alice multiples $p$ and $q$ together to get another number, $N$
(capitalized because it's really big). In this case, $N=17\times11=187$.
She now picks another number $e$, the \textbf{encryption key}, with
the property that $e^{-1}$ exists mod $\phi(pq)$, i.e. $\gcd(e,\phi(pq))=1$.
In this case, $\phi(pq)=(17-1)(11-1)=160$, and Alice chooses $e=7$,
which has the property that $\gcd(7,160)=1$. In general, $\phi(pq)=(p-1)(q-1)$
as above.
\item Alice can now publish $e$ and $N$ in something akin to a telephone
directory (but online). Since these two numbers are necessary for
encryption, they must be available to anyone who wants to send Alice
a message. Together, the numbers $e$ and $N$ are called the \textbf{public
key. }(Everyone must have their own value of $N$ if they want to
receive messages, since $N$ depends on $p$ and $q$.)
\item To encrypt a message $M$, we convert the message's letters to ASCII
(underneath, the computer is doing everything in binary). Suppose
Bob wants to send Alice a simple kiss, $M=X=88$. Now, similar to
Diffie-Hellman, the message $M$ is encrypted using the agreed-upon
equation
\[
C\equiv M^{e}\mod N
\]
where $C$ is the ciphertext number and $M$ is the message. To encrypt
this message, Bob looks up Alice's public key, and discovers that
$N=187$ and $e=7$. This provides Bob with the encryption formula
required to send messages to Alice: $C\equiv M^{7}\mod187$. Bob plugs
his kiss $M=88$ into this equation to get
\[
C\equiv88^{7}\mod187.
\]
Bob (really his computer) uses the modular exponentiation algorithm
to compute $88^{7}\mod187$:
\begin{align*}
88^{7}\mod187 & =88^{4}\times88^{2}\times88^{1}\mod187\\
88^{1} & =88\mod187\\
88^{2} & =7744\equiv77\mod187\\
88^{4} & =59969536\equiv132\mod187\\
88^{7} & \equiv88\times77\times132=894432\equiv11\mod187.
\end{align*}
Bob now sends the ciphertext, $C=11$, to Alice.
\item We know that modular exponentiation is a one-way function, so it is
very difficult to work backwards from $C=11$ and recover the original
message $M$. Hence, Eve cannot decipher the message. However, Alice
can decipher the message because she has special information: the
values of $p$ and $q$. She computes
\[
d=e^{-1}\mod\phi(N)
\]
using something called the Extended Euclidean Algorithm (or a modular
inverse table on a computer). For example, in this case $d=23\mod(16\times10)=23\mod160$.
\item To decrypt the message, Alice uses the formula 
\begin{align*}
M & \equiv C^{d}\mod N\\
 & \equiv11^{23}\mod187\\
 & \equiv11^{1}\times11^{2}\times11^{4}\times11^{16}\mod187\\
 & \equiv11\times121\times55\times154\mod187\\
 & \equiv88\mod187.
\end{align*}
Now Alice has deciphered Bob's kiss, $X=88$.
\end{enumerate}
\end{lyxalgorithm}

\begin{itemize}
\item Among other things, what's great about this algorithm is that anyone
can send a message to Alice using her equation $C\equiv M^{7}\mod187$,
but this information is not enough to decipher the message without
knowing the values of $p$ and $q$. And only Alice knows these values!
\item The reason that the decipherment works if Alice knows $p$ and $q$
is Euler's Theorem: $d=e^{-1}$ is the multiplicative inverse of $e\mod\phi(pq)$,
meaning that $ee^{-1}\equiv1\mod\phi(pq)$, hence $ee^{-1}$ is $1$
more than a multiple of $\phi(pq)$. Call this multiple $k\phi(pq)$;
then $ee^{-1}=1+k\phi(pq)$. Now Alice computes
\begin{align*}
C^{d}\mod N & =(M^{e})^{d}\mod N\\
 & \equiv M^{ee^{-1}}\mod pq\\
 & \equiv M^{1+k\phi(pq)}\mod pq\text{ for some integer }k\\
 & \equiv M\times(M^{\phi(pq)})^{k}\mod pq\\
 & \equiv^{Euler}M(1)^{k}\mod pq\\
 & \equiv M\mod pq.
\end{align*}
\end{itemize}
\begin{example}
Alice publishes her public kerys $e=7$ and $pq=77$ for all to see.
To send the message $M=25<pq$ to Alice, Bob computes 
\[
C\equiv M^{e}\mod pq=25^{7}\mod77\equiv53\mod77
\]
and sends it to Alice. Alice uses her private key, $d=43$, to compute
\begin{align*}
C^{d}\mod pq & =53^{43}\mod77\\
 & \equiv25\mod77\\
 & =M.
\end{align*}
If Eve intercepts $C=53$, she can decipher it only if she can solve
$M^{7}\mod77\equiv53$ for $M$. But this is the discrete logarithm
problem, which is too hard to solve if we use big enough numbers!
\end{example}

\begin{xca}
With the person across from you, transmit the message ``25'' using
RSA and the following keys. Make sure to specify which parts of your
communication are public and private!
\begin{enumerate}
\item $p=5$, $q=7$, $e=5$
\item $p=3$, $q=5$, $e=3$
\end{enumerate}
\end{xca}


\section{Why does RSA work?}
\begin{itemize}
\item The trapdoor functions of RSA hinge entirely on Euler's Theorem.
\item But why is Euler's Theorem true?
\end{itemize}
\begin{xca}
Let $a=3,n=7$. Compute (splitting the work up in groups) $a\mod n,2a\mod n,3a\mod n,\dots,6a\mod n$.
\begin{itemize}
\item $3,6,2,5,1,4$, in that order. Notice that $3^{-1}=5\mod7$ since
$3\times5=15\equiv1\mod7$. 
\end{itemize}
\end{xca}

\begin{itemize}
\item This is true in general if $a,n$ are two numbers that don't share
any factors other than $1$:
\end{itemize}
\includegraphics[scale=0.75]{\string"Old Cryptography Notes/pasted190\string".png}

Now

\includegraphics[scale=0.75]{\string"Old Cryptography Notes/pasted185\string".png}
\begin{xca}
Compute $3^{\phi(7)}\mod7$ by multiplying $(3\cdot1)(3\cdot2)(3\cdot3)(3\cdot4)(3\cdot5)(3\cdot6)$
and verifying the result is the same as $(1)(2)(3)(4)(5)(6)\mod7$.
Since the former is $3^{\phi(7)}(1\cdot2\cdot3\cdot4\cdot5\cdot6)\mod7$,
this means 
\begin{align*}
3^{\phi(7)}(1\cdot2\cdot3\cdot4\cdot5\cdot6)\equiv(1\cdot2\cdot3\cdot4\cdot5\cdot6)\mod7.
\end{align*}
What is the inverse of $(1\cdot2\cdot3\cdot4\cdot5\cdot6)\mod7$?
Multiply both sides by this inverse. The left side becomes $3^{\phi(7)}$.
What does the right side become? What does this say about $3^{\phi(7)}\mod7$?
\end{xca}

\includegraphics[scale=0.75]{\string"Old Cryptography Notes/pasted186\string".png}

\section{RSA in the real world}
\begin{itemize}
\item The message $M$ could be a private number, like a credit card number,
that one party wishes to send to another, or $M$ could be an encoded
message or part of a message (in ASCII).
\item Encrypting data using RSA is relatively slow, so if Alice and Bob
want to exchange a lot of data (a lot of $M$'s), then it might be
wise to have $M$ be a shared key for something like a one-time pad
instead.
\item Finally, RSA can be used to generate digital signatures. If Bob sends
a message to Alice, how does Alice know that Bob really sent it?
\item One thing that Bob can do is encrypt his ``signature'' with his
own \emph{private} key, and when Alice receives his message, she can
decrypt it using Bob's \emph{public }key (since $(M^{e})^{d}\equiv M$,
it's also true that $(M^{d})^{e}\equiv M$) because multiplication
is commutative).
\end{itemize}
\begin{example}
Bob sends a surprising message to Alice: ``Alice, I've decided to
major in math. It's the coolest! Bob ($125010690$)

Bob knows that Alice won't believe he actually sent the message, so
he digitally signed it by enciphering his name (in ASCII) using his
own private key. Alice finds that Bob's public keys are $e=1234567891$
and $pq=176391331$, and she computes
\begin{align*}
152010690^{e}\mod pq & =125010690^{1234567891}\mod176391331\\
 & \equiv66111098\mod176391331.
\end{align*}
Since the decrypted digital signature ($66$ $111$ $098$) has an
ASCII equivalent of \emph{Bob}, Alice is sure that Bob actually sent
the message.
\end{example}

\begin{xca}
{[}slide{]}
\end{xca}

\includegraphics[scale=0.75]{\string"Old Cryptography Notes/pasted189\string".png}
\end{document}
