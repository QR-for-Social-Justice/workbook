%% LyX 2.3.6.2 created this file.  For more info, see http://www.lyx.org/.
%% Do not edit unless you really know what you are doing.
\documentclass[oneside,english]{amsart}
\usepackage[T1]{fontenc}
\usepackage{geometry}
\geometry{verbose,tmargin=1in,bmargin=1in,lmargin=1in,rmargin=1in}
\usepackage{babel}
\usepackage{amsthm}
\usepackage{graphicx}
\usepackage[unicode=true]
 {hyperref}

\makeatletter

%%%%%%%%%%%%%%%%%%%%%%%%%%%%%% LyX specific LaTeX commands.
%% Because html converters don't know tabularnewline
\providecommand{\tabularnewline}{\\}

%%%%%%%%%%%%%%%%%%%%%%%%%%%%%% Textclass specific LaTeX commands.
\numberwithin{equation}{section}
\numberwithin{figure}{section}
\theoremstyle{plain}
\newtheorem{thm}{\protect\theoremname}
\theoremstyle{plain}
\newtheorem{question}[thm]{\protect\questionname}
\theoremstyle{definition}
\newtheorem*{xca*}{\protect\exercisename}
\theoremstyle{definition}
\newtheorem{xca}[thm]{\protect\exercisename}
\theoremstyle{definition}
\newtheorem{example}[thm]{\protect\examplename}

\makeatother

\providecommand{\examplename}{Example}
\providecommand{\exercisename}{Exercise}
\providecommand{\questionname}{Question}
\providecommand{\theoremname}{Theorem}

\begin{document}
\title{WCSAM 206 Day 2 - Ancient Ciphers}
\author{Kenan Ince}
\maketitle

\section{Preliminaries}
\begin{enumerate}
\item Check reading notes for \emph{The Code Book }p1-14
\begin{enumerate}
\item Steganography (Xerxes hiding a message underneath wax on a piece of
wood; scytales) vs. cryptography (making it less likely that brute
force will work to decrypt a message)
\begin{enumerate}
\item Note that we used a keyword cipher with key OLIVER to encrypt the
poem from last class (still a monoalphabetic substition cipher)
\end{enumerate}
\end{enumerate}
\end{enumerate}

\subsection*{Decoding cryptography jargon}
\begin{itemize}
\item \textbf{Cryptography }(from Greek \emph{kryptos-}, meaning ``hidden'')
is secret communication by hiding the meaning of a message, not its
existence. The process of hiding the meaning of a message is called
\textbf{encryption.} The recipient of the message has to \textbf{decrypt
}or \textbf{decipher} the message in order to read it. Often the method
of encryption relies on a \textbf{key}, some special number(s) or
word(s) that only the sender and recipient know.
\item Before computers, encryption methods were relatively simple, not explicitly
mathematical, and often not very secure. Encryption methods commonly
used today tend to use very sophisticated mathematics. Today, we all
use cryptographic methods when we use our cell phones or make online
purchases.
\item \textbf{Cryptanalysis} is the study of cryptographic algorithms with
the intent of recovering secret messages without knowing the secret
key. Example: enemy governments, spies.
\item A \textbf{cipher} is a set of steps (or \textbf{algorithm}) for encrypting
a message, or \textbf{plaintext}, into apparently unintelligible \textbf{ciphertext.}
\end{itemize}

\section*{Types of ciphers}
\begin{itemize}
\item There are two basic tools that can be used in encryption algorithms:
\textbf{transposition} (rearrange the characters) and \textbf{substitution}
(replacing characters with other characters).
\end{itemize}
\begin{question}
Do you know any transposition or substitution ciphers?
\end{question}

\begin{itemize}
\item Now, time for a group exercise:
\end{itemize}
\begin{xca*}
{[}have students use Coursepacks, additional paper{]}

\includegraphics{\string"Old Cryptography Notes/pasted1\string".png}
\end{xca*}
\begin{itemize}
\item Another form of transposition is used in the first ever military cryptographic
device, the Spartan \emph{scytale} ($\sigma\kappa\nu\tau\alpha\lambda\eta$):
{[}hand out several different diameter tubes, one to each group; use
pencils{]}
\item Now, take a piece of paper, wrap it around your tube, and write an
{*}plaintext{*} message on it. Then unwrap it. See how it's encrypted?
Pass it to the group to your right and see if they can decipher it
\emph{without showing them the tube you wrapped it around.}
\end{itemize}

\section*{The Caesar cipher}

\includegraphics{\string"Old Cryptography Notes/pasted2\string".png}

The Caesar cipher is an example of a \textbf{monoalphabetic substitution
cipher}, in which every character is replaced by some other character. 

\includegraphics{\string"Old Cryptography Notes/pasted3\string".png}
\begin{xca}
{[}slide{]} Substitution Ciphers PDF (Coursepack p94-98) \href{https://drive.google.com/file/d/1LcZXss5WHM9_ZguPoo-tY7SGt41IEGRB/view}{Lesson plan}

How did you decrypt substitution ciphers in the worksheet?
\end{xca}


\section{Decrypting simple shift ciphers with known key}
\begin{itemize}
\item Here's a quicker way to decrypt shift ciphers.
\item Let's say we're trying to decrypt ``bpiwtbpixrpxhiwtqthiegdvgpb''
(``Mathematica is the best program'') with $k=15$. That means that
we're going to shift every letter \textbf{back} by 15 letters.
\item Try shifting ``b'' back by 15 letters. 
\item Now we have to do it for the rest of the code. It's going to take
a while, right? Any ideas on how to speed this up?
\item Well, there are 26 letters in the alphabet. Shifting ``b'' back
by 15 gives us: 
\[
a,z,y,x,w,v,u,t,s,r,q,p,o,n,m
\]
\item Notice this is the same result as you get when shifting ``b'' \textbf{forward}
by 11. The alphabet is cyclical, so shifting back by 15 is the same
as shifting forward by 11.
\item Now let's shift ``p'' forward by 11. 
\item This is where it's beneficial to have a \textbf{letter-number correspondence
sheet}; number the letters in increasing order starting with $a=0$.

\includegraphics{\string"Old Cryptography Notes/pasted42\string".png}
\item Well, ``p'' is the 15th letter in the alphabet. Adding 11 to 15
gives 26. There's no 26th letter of the above alphabet, right? But
the 25th letter is ``z'', so the ``26th letter'' should be a shift
of one letter from ``z''. That gives ``a''! \emph{What we're really
saying here is that, in the sense of the alphabet, $26=0$. We'll
learn more about this later.}
\item Keep going\textendash it's straightforward for p and i. For w, we
get 33, which is $26+7$. So it's a shift of 7 from ``z'', which
is the same as the 7th letter of the alphabet, which is \textbf{h}. 
\end{itemize}
\begin{tabular}{|c|c|c|c|c|c|c|c|c|c|c|c|c|c|c|c|c|c|c|c|c|c|c|c|c|c|c|c|}
\hline 
Ciphertext & b & p & i & w & t & b & p & i & x & r & p & x & h & i & w & t & q & t & h & i & e & g & d & v & g & p & b\tabularnewline
\hline 
\hline 
Ciphertext number & 1 & 15 & 8 & 22 & 19 & 1 & 15 & 8 & 23 & 17 & 15 & 23 & 7 & 8 & 22 & 19 & 16 & 19 & 7 & 8 & 4 & 6 & 3 & 21 & 6 & 15 & 1\tabularnewline
\hline 
Plaintext number & 12 & 1 & 19 & 33 & 4 &  &  &  &  &  &  &  &  &  &  &  &  &  &  &  &  &  &  &  &  &  & \tabularnewline
\hline 
Plaintext & \textbf{m} & \textbf{a} & \textbf{t} & \textbf{h} & \textbf{e} & ...you finish this. &  &  &  &  &  &  &  &  &  &  &  &  &  &  &  &  &  &  &  &  & \tabularnewline
\hline 
\end{tabular}
\begin{example}
Note: somehow $33=7$ in the alphabet! This is like thinking of the
alphabet as a clock {[}draw{]} with $26$ numbers on it, going from
$0$ on the top to $25$. Where is 33:00 on this clock? It's aka 7:00!
\end{example}

\begin{xca}
\textbf{(Reading question) }We say two numbers $a$ and $b$ are \textbf{congruent
modulo $n$}, written $a\equiv b\mod n$,\textbf{ }if one of the following
equivalent statements is true:
\begin{itemize}
\item $n$ evenly divides $b-a$ (with no remainder), written $n\mid(b-a)$.
\item $a$:00 and $b$:00 would be the same time on a clock with $n$ numbers
(where $n=0$).
\item $a$ and $b$ have the same remainder upon division by $n$.
\item subtracting some multiple of $n$ from $a$ yields $b$ or vice versa.
\end{itemize}
\textbf{Reducing a number $a\mod n$ }means the process of finding
the remainder when you divide $a$ by $n$.
\begin{enumerate}
\item (1.5) Reduce the following numbers modulo 16: 
\begin{enumerate}
\item $27$ has remainder $11$ when divided by $16$, so $27\equiv11\mod16$.
\item $544$: divide $544$ by $16$ to get $34$, remainder $0$. So $544\equiv0\mod16$.
\item $32\equiv0\mod16$ because $\frac{32}{16}=2$, remainder $0$.
\end{enumerate}
\end{enumerate}
\end{xca}


\section*{Python time: work on Codecademy lessons}
\begin{itemize}
\item Demonstrate how to log in and start the lessons
\end{itemize}

\section*{Reflection}
\begin{enumerate}
\item What were the two most important concepts from today's class?
\item What were two things you didn't fully understand?
\item What were your two favorite things we did in class today?
\end{enumerate}

\section*{Additional exercises}

0.1, 6, 7, 8

\includegraphics{\string"Old Cryptography Notes/pasted98\string".png}

\includegraphics{\string"Old Cryptography Notes/pasted123\string".png}
\end{document}
