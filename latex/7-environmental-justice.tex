%% LyX 2.3.6.2 created this file.  For more info, see http://www.lyx.org/.
%% Do not edit unless you really know what you are doing.
\documentclass[oneside,english]{amsart}
\usepackage[LGR,T1]{fontenc}
\usepackage{geometry}
\geometry{verbose,tmargin=1in,bmargin=1in,lmargin=1in,rmargin=1in}
\usepackage{amstext}
\usepackage{amsthm}
\usepackage{graphicx}

\makeatletter

%%%%%%%%%%%%%%%%%%%%%%%%%%%%%% LyX specific LaTeX commands.
\DeclareRobustCommand{\greektext}{%
  \fontencoding{LGR}\selectfont\def\encodingdefault{LGR}}
\DeclareRobustCommand{\textgreek}[1]{\leavevmode{\greektext #1}}
\ProvideTextCommand{\~}{LGR}[1]{\char126#1}


%%%%%%%%%%%%%%%%%%%%%%%%%%%%%% Textclass specific LaTeX commands.
\numberwithin{equation}{section}
\numberwithin{figure}{section}
\theoremstyle{plain}
\newtheorem{thm}{\protect\theoremname}
\theoremstyle{plain}
\newtheorem{question}[thm]{\protect\questionname}
\theoremstyle{definition}
\newtheorem{xca}[thm]{\protect\exercisename}
\theoremstyle{definition}
\newtheorem{example}[thm]{\protect\examplename}
\theoremstyle{definition}
\newtheorem{defn}[thm]{\protect\definitionname}
\theoremstyle{remark}
\newtheorem{note}[thm]{\protect\notename}

\makeatother

\usepackage{babel}
\providecommand{\definitionname}{Definition}
\providecommand{\examplename}{Example}
\providecommand{\exercisename}{Exercise}
\providecommand{\notename}{Note}
\providecommand{\questionname}{Question}
\providecommand{\theoremname}{Theorem}

\begin{document}
\title{WCSBS 220 7 - Environmental Justice \& Environmental Racism}
\maketitle
\begin{itemize}
\item Anonymous course feedback forms: multiple people want more time spent
on assignments; one person wants less. So I'll try and have more full-class
discussions about worksheets.
\end{itemize}

\section{Reading Question}
\begin{itemize}
\item A recent article in the Proceedings of the National Academy of Sciences
of the United States of America put the impact of pollution on communities
of color into a full racial-equity framework (2020, DOI: 10.1073/pnas.1818859116).
On average, the study found that non-Hispanic white people experience
a \textquotedblleft pollution advantage\textquotedblright{} of 17\%
less air pollution exposure than is caused by the goods and services
they consume. In contrast, Blacks and Hispanics bear a \textquotedblleft pollution
burden\textquotedblright{} of 56\% and 63\% excess exposure, respectively,
relative to the pollution caused by their consumption.
\item In April 2020, the Harvard T.H. Chan School of Public Health published
statistics linking air pollution to higher COVID-19 death rates. Looking
at 3,000 counties, the preprint, currently in peer review, found that
someone who lives for decades in a county with high levels of fine
particulate pollution is 8\% more likely to die from COVID-19 than
someone who lives in a region that has just one unit (1 \textgreek{m}g/m3)
less of such pollution (medRxiv 2020, DOI: 10.1101/2020.04.05.20054502).
\item These findings follow a 2017 report by the National Association for
the Advancement of Colored People (NAACP) and the Clean Air Task Force
finding that African Americans are exposed to 38\% more air pollution
than white people and are 75\% more likely to live near toxic pollution
than the rest of the American population.
\end{itemize}
.
\begin{question}
What questions do you have about climate change and extinction?
\begin{itemize}
\item What is climate? What is weather? How can climate change be happening
when SLC still sometimes sets low temperature records?
\item What's causing climate change?
\item Does climate change only mean global warming? 
\item How many scientific studies have been done in support of climate change,
and how accurate are they?
\item To determine whether climate is an issue, should we measure temperatures
around the world evenly, or should we measure only in certain places
(sampling?)
\end{itemize}
\end{question}

\begin{xca}
\textbf{(RQ) }Bill McKibben is the leader of the anti-carbon emission
group 350.org and an American environmentalist author focused on the
impact of climate change. Watch McKibben's video below, then answer
the following questions in a text box or document. In addition, bring
copies of your responses to class on the date below for discussion.
\begin{enumerate}
\item What is the significance of each of the three numbers that McKibben
mentions? 
\item Does the video spin the numbers at all? Explain. If so, what tricks
does the video use to spin the numbers? 
\item How would you spin the number(s) in the article to make them seem
big? Small? 
\item Fact-check McKibben's numbers and rate the video's statements about
each of the three numbers on two scales: 
\begin{enumerate}
\item How true is the statement? (0 = completely false, 4 = completely true) 
\item How misleading is the statement? (0 = not at all, 4 = totally misleading)
(In other words, when you look at the underlying data, does it support
the video's point or not?)
\end{enumerate}
\end{enumerate}
\end{xca}


\section{What is environmental justice?}
\begin{itemize}
\item Protecting the environment is sometimes viewed as a luxury -{}- something
people care about only when they have plenty of leisure time and disposable
income. In practice, low-income communities and racialized groups
often bear the most severe consequences of environmental degradation
and pollution.
\item Our discussion is centered on \textbf{environmental justice}: the
recognition that minority and low-income communities often bear a
disproportionate share of environmental costs \textendash{} and the
perception that this is unjust.
\begin{itemize}
\item Across the United States, poor and PoC neighborhoods bear an unequal
burden from hazardous facilities and waste sites. This pattern is
evident nationally as well as on the state and local level.
\item Pollution is unequally distributed across the country; it is also
distributed unequally within individual states, within counties, and
within cities. 
\item Hazardous waste sites, municipal landfills, incinerators, and other
hazardous facilities are disproportionately located in poor and PoC
neighborhoods.
\item Environmental contaminants may also be carried long distances, affecting
communities far from their origin. Some might argue that this is just
the way the market works, since richer communities or nations can
afford better environmental protection. 
\item Others contend that the issue involves broader questions of human
rights and fundamental justice, requiring national policies, or international
agreements, to modify the workings of unregulated markets.
\end{itemize}
\item \textbf{Environmental racism }is the system by which people of color
bear a disproportionate share of environmental costs.
\item \textbf{Environmental classism} is the system by which poorer people
bear a disproportionate share of environmental costs.
\end{itemize}
\textbf{Key Question}: How does one collect data that allow us to
evaluate the state of environmental justice and environmental racism?
What and how should we measure to determine the extent of environmental
injustice?
\begin{example}
Collect data from the class on the question ``do you live within
$5$ miles of a refinery or hazardous waste?'' then point out possible
response error and convenience sample.
\end{example}

\begin{xca}
Design your own study to measure the extent of environmental injustice
in some way, then share with the class. Make a table of studies: sample
surveys and experiments, by individuals, variables, how well variables
are a proxy for what's being measured, whether they're SRS's/controlled,
population, sampling strategy, explanatory/response variables, etc
\end{xca}


\section{What is statistics?}
\begin{itemize}
\item Statistics is the science of data. We could almost say ``the art
of data''.
\end{itemize}

\subsection{Central themes of statistics}
\begin{itemize}
\item Data beat anecdotes: 
\begin{itemize}
\item An \textbf{anecdote }is a striking story that sticks in our minds
exactly because it is striking.
\item \textquotedbl It's easy to lie with statistics, but it's even easier
to lie without them.\textquotedbl{} -Frederick Mosteller 
\item Does living near power lines cause leukemia in children? vs. mother
whose child has leukemia
\end{itemize}
\item Where the data come from is important: what kind of errors might creep
in? 
\begin{itemize}
\item Homeopathic remedies e.g. Airborne produce benefits, but no more than
a placebo (sugar pill)
\end{itemize}
\item Beware the \textbf{lurking variable, }a third variable that affects
all of the variables you're measuring
\begin{itemize}
\item Correlation does not imply causation 
\item Higher education correlates with higher income
\item Ice cream sales correlate with increased crime rate
\end{itemize}
\item Variation is everywhere 
\begin{itemize}
\item Numbers are a shorthand for a range of values due to variation 
\item $98.6^{\circ}$ is not the exact human body temperature; more accurate
to say it's $98^{\circ}$.
\end{itemize}
\item Conclusions are not certain 
\begin{itemize}
\item Variation throws a wrench in things 
\item Do mammograms reduce the risk of dying of breast cancer?
\begin{itemize}
\item ``We are $95\%$ confident that mammography reduces the risk of dying
of breast cancer by between $17$ and $34$ percent.''
\end{itemize}
\end{itemize}
\item Data reflect social values 
\begin{itemize}
\item Much of the difference in reported suicide rates between nations is
due to social attitudes.
\begin{itemize}
\item Where suicide is stigmatized, deaths are more often reported as accidents.
\item Japanese culture has a tradition of honorable suicide in response
to shame, leading to better reporting of suicides
\end{itemize}
\item There is no such thing as perfect objectivity.
\end{itemize}
\end{itemize}
\begin{xca}
\textbf{{[}Reading Question{]} }Read the Preface (p. viii-ix) and
the Executive Summary (p. x-xv) of the report Toxic Wastes and Race
at Twenty: 1987-2007 authored by Robert Bullard et al., then answer
the following questions in your own words in complete sentences. Turn
your answers in online and bring copies to class on the date below
for discussion.
\begin{enumerate}
\item What were the findings of the original Toxic Wastes and Race in the
United States report? 
\begin{enumerate}
\item In the original report, it was found that race played more into predicting
where toxic waste facilities would be located in the U.S. than other
variables like home value, household income, and waste levels generated
in industrial areas. 
\end{enumerate}
\item State three additional findings of the updated Toxic Wastes and Race
at Twenty report. 
\begin{enumerate}
\item Of the estimated 9,000,000 people in the U.S. who live within 3 km
of commercial hazardous waste facilities, 5.1 million of them are
people of color living in neighborhoods near one or more of these
facilities.
\item In 44 states, there are disproportionately high percentages of people
of color in \textquotedblleft host neighborhoods\textquotedblright{}
where toxic waste facilities are located. That is to say that between
a host area and a non-host area, there is a much higher percentage
of people of color in the host areas than there is in non-host areas.
Michigan has the largest difference: 66\% of host neighborhoods are
made up of people of color, while 19\% is made up of people of color
in non-host neighborhoods. 
\item 105/149 (70\%) metropolitan areas that host toxic waste facilities
have disproportionately large populations of people of color, and
in 46 of these metropolitan areas, the majority of host neighborhoods
are made up of people of color.
\end{enumerate}
\item Why might industries disproportionately dump toxic waste in areas
with high populations of people of color? 
\begin{enumerate}
\item In the words of the report, polluting industries want to take the
\textquotedblleft path of least resistance\textquotedblright{} in
dumping toxic waste. The perception may be that people of color will
not fight back against the pollution of their neighborhoods with toxic
waste because they fear putting their jobs and economic stability
at risk. 
\item People of color are generally more likely to experience poverty, and
to have a more difficult time securing high-paying jobs because of
racist hiring practices/institutionalized racism in many job markets.
The heads of industries may assume that because of this, people of
color will resist less- especially because they are already dealing
with many other injustices. 
\end{enumerate}
\item In your own words, describe one recommendation of the report and respond
to it. Do you agree or disagree that this recommendation will promote
greater justice? Why or why not? 
\begin{enumerate}
\item One of the recommendations of the report was for Congress to create
laws and regulations that would force industries to use production
systems that reduce waste and eliminate as much pollution as possible,
and to create incentives for industries to reuse, recover, and recycle
waste. 
\item I think that this type of legislation would be a step in the right
direction for environmental justice. People of color are disproportionately
affected by toxic waste disposal, so regulating the waste and pollution
of industry would directly impact neighborhoods with higher populations
of people of color. These regulations would also be instrumental in
reducing overall global pollution levels. 
\end{enumerate}
\item Protecting the environment is sometimes viewed as a luxury -{}- something
people care about only when they have plenty of leisure time and disposable
income. How would you respond to this viewpoint after reading the
report?
\begin{enumerate}
\item First of all, my opinion is that the responsibility of protecting
the environment primarily falls into the hands of the government and
big polluting industries. There should be greater restriction and
regulation of waste produced by industries. Individual action can\textquoteright t
make an impact significant enough to protect the environment as long
as big industry continues to cause pollution and environmental harm
at such high rates. 
\item That being said, I think that it is true that protecting the environment
can be a luxury. It can cost a lot of money to recycle, or to drive
a fuel-efficient, environmentally-friendly car. Also, wealthier people
have the luxury of being able to move out of a neighborhood in which
the local environment is toxic/hazardous. Someone who lives in an
area around a refinery or some other toxic waste facility isn\textquoteright t
going to make much change to the already-toxic environment by carpooling
or recycling, because the waste facility is creating much more environmental
damage than any single human could be capable of. 
\end{enumerate}
\end{enumerate}
\end{xca}


\section{Where do data come from?}

\subsection{What should we measure?}
\begin{itemize}
\item \textbf{Individuals} are the objects described by a data set; may
be animals or things
\item {[}slide{]} A \textbf{variable} is any characteristic of an individual.
A variable can take on different values for different individuals.
\end{itemize}
\includegraphics{\string"../../DATA 150 Data and Society/DATA 150 Class Notes/pasted1\string".png}
\begin{example}
What are the individuals in this data set? What variables are in this
data set? Which take on numerical values?
\end{example}

\begin{itemize}
\item Consider the following example of the difficulties in choosing what
data to collect.
\end{itemize}
\begin{xca}
Researchers spent lots of time and money weighing the stuff put out
for recycling in two neighborhoods in a California city; call them
Upper Crust and Lower Mid. The Upper Crust households contributed
more pounds per week on average than the folk in Lower Mid. 
\begin{enumerate}
\item What are the individuals in this data set?
\item What are the variables? Which take on numerical values?
\item Can we say that the rich are more serious about recycling?
\begin{enumerate}
\item No; they recycle more glass wine bottles and the poor more plastic
beer and soda cans.
\end{enumerate}
\item Weight is not a good measure of the participation of households in
different neighborhoods in a city recycling program. What variables
would you measure in its place?
\end{enumerate}
\end{xca}


\subsection{Observational studies}
\begin{defn}
An \textbf{observational study} observes individuals and measures
variables of interest but does not attempt to influence the response.
An \textbf{explanatory variable }is a variable that we think causes
changes in another (for example, wealth in the study above). A \textbf{response}
or \textbf{response variable }is a variable that measures the outcome
or result of a study (for example, how much people recycled). The
purpose of an observational study is to describe some group or situation.
\end{defn}

\begin{xca}
Could we do an experiment to measure whether living near fracking
wells was correlated with health issues? How ethical would this experiment
be?
\end{xca}

\begin{example}
Researchers compared $638$ children who had leukemia and $620$ who
did not. They measured the magnetic fields in the children's bedrooms,
other rooms, and the front door. The study found that no risk of leukemia
due to power lines stood out from the play of chance that distributes
leukemia cases across the landscape. Critics continue to argue that
the study failed to measure some important variables or that the children
studied don't fairly represent all children.
\end{example}

\begin{xca}
\textbf{{[}RQ{]} }Hydraulic fracturing (also known as \textquotedblleft fracking\textquotedblright )
is a method to extract natural gas reserves in shale deposits thousands
of feet underground. Fracking activity has increased substantially
in recent years, both in the United States and abroad. However, there
is very little peer-reviewed research about the affects fracking has
public health. There is growing concern that fracking could contaminate
local water supplies and ambient air pollution, ultimately resulting
in negative public health effects. The article \emph{Proximity to
Natural Gas Wells and Reported Health Status: Results of a Household
Survey in Washington County, Pennsylvania} is an \textquotedblleft attempt
to assess the relationship between household proximity to natural
gas wells and reported health symptoms.\textquotedblright{}

Before class: please read the article in its entirety, and answer
the following 6 questions. We\textquoteright ll begin next class by
discussing these questions, so complete and well-formed answers are
appreciated! Turn in your work online, then bring a copy to class
on the date below for discussion.
\begin{enumerate}
\item \textbf{Human Health Impact}: Among the studies cited in this paper
are two convenience sample surveys (Steinzor et al. 2013 and Ferrar
et al. 2013). What is a convenience sample survey? What are some problems
with this type of survey? 
\begin{enumerate}
\item A \textbf{convenience sample survey }is a survey whose subjects are
those people who were easiest for the surveyor to interview.
\item Problems with convenience sample surveys usually stem from the fact
that those who are easiest for the surveyor to interview, for example
college students, are not representative of the population as a whole.
For example, they may be wealthier and more liberal.
\end{enumerate}
\item \textbf{Selection and recruitment of households}: How do the authors
find their participants to survey? Compare this method to a convenience
study \textendash{} which yields more meaningful results, and why? 
\begin{itemize}
\item ``In this study we focused on Washington County in southwestern Pennsylvania,
an area of active natural gas drilling. We selected a contiguous set
of 38 rural townships within the center of Washington County as our
study site in order to avoid urban areas bordering Pittsburgh, which
would be unlikely to have ground-fed water wells, and areas near the
Pennsylvania border, which might be influenced by gas wells in other
states.''
\item Using ArcGIS Desktop 10.0 software (ESRI, Inc., Redlands, CA), we
randomly selected 20 geographic points from each of 38 contiguous
townships in the study county (Figure 1). We identified an eligible
home nearest to each randomly generated sampling point, and visited
each home to determine which households were occupied and had ground-fed
water wells. We selected households with ground-fed water wells to
assess possible health effects related to water contamination. From
the original 760 points identified (i.e., 20 points in each of the
38 townships), we excluded 12 duplicate points and 64 points found
not to correspond to a house structure (see Supplemental Material,
Figure S1). 
\end{itemize}
\item \textbf{Administration of survey at residence}: The authors note that
\textquotedblleft survey personnel were not aware of the mapping results
for gas well proximity to the households being surveyed.\textquotedblright{}
Why is this statement an important thing to mention? How does this
relate to the experimental design element known as blinding? (More
info about blinding here.) 
\begin{itemize}
\item If survey personnel, for example, thought that proximity to natural
gas wells was correlated with negative health effects, the tone of
their questions and their behavior may lead the survey respondent
to consciously or unconsciously modify their responses to fit this
narrative. After all, the surveyer is affiliated with a scientist,
so they must know what they're talking about!
\item A \textbf{blind }or \textbf{blinded study} is a scientific study (usually
an experiment) in which information about the test is masked (kept)
from the participant, to reduce or eliminate bias, until after a trial
outcome is known. It is understood that bias may be intentional or
subconscious, thus no dishonesty is implied by blinding. If both tester
and subject are blinded, the trial is called a \textbf{double-blind
experiment}.
\item In this example, since there are no control or experimental groups,
the only person who can be blinded is the experimenter. This is \textbf{not
}a double-blind study, just a blinded study.
\end{itemize}
\item \textbf{Discussion}: Describe any potential sources of bias in this
study, and explain how such bias could influence or affect the conclusions
of the study. 
\begin{enumerate}
\item \textbf{Non-sources of bias: }
\begin{enumerate}
\item ``This association persisted even after adjusting for age, sex, smokers
in household, presence of animals in the household, education level,
work type, and awareness of environmental risks...
\item Strengths of the study included the larger sample size compared with
previously published surveys, and the random method of selecting households
using geographic information system methodology, which reduces the
possibility of selection bias (although only a subset of households,
those with ground-fed water supply, were sampled).''
\item When a study attempts to diagnose its shortfalls, that's a good sign!
\end{enumerate}
\item \textbf{Limitations/potential sources of bias}: 
\begin{enumerate}
\item reliance on self-report of health symptoms {[}why is this bad?{]},
which could lead to respondents who are aware of potential health
impacts of fracking to overreport their symptoms (\textbf{recall bias})
\item We did not collect data on whether individuals were receiving financial
compensation for gas well drilling on their property, which could
have affected their willingness to report symptoms. It is possible
that differential refusal to participate could have introduced potential
for selection bias; for example, individuals who were receiving compensation
for gas drilling on their property might be less willing to participate
in the survey. 
\item We found instead that the refusal rate, though $<25\%$ overall, was
higher among households farther from gas wells, suggesting that such
households may have been less interested in participating because
they had less awareness of hazards. The study questionnaire did not
include questions about natural gas extraction activities, in order
to reduce awareness bias. At the same time, it is likely that household
residents were aware of gas drilling activities in the vicinity of
households; and the fact that reported environmental awareness by
respondents was associated with the prevalence of all groups of reported
health symptoms suggests a correlation between heightened awareness
of health risks and reported health conditions. 
\begin{question}
Why is it important that the \textbf{refusal }or \textbf{nonresponse
rate }be low? Why does the study carefully investigate the demographics
of those who did not respond?
\begin{itemize}
\item Because removing those who don't respond from the study may lead to
bias\textendash for instance, say you're conducting a political survey
where no Trump supporters respond because they think that all media
surveys are biased.
\item By analyzing the demographics of those who don't respond, we at least
get some idea of the direction in which our survey is skewed.
\end{itemize}
\end{question}

\end{enumerate}
\end{enumerate}
\item \textbf{Discussion}: The authors state that their \textquotedblleft use
of gas well proximity\dots{} was an indirect measure of potential
water or airborne exposures.\textquotedblright{} Unpack that statement
\textendash{} what, precisely, would the authors like to measure about
gas wells? What do they end up measuring instead? 
\begin{itemize}
\item The cause of potential illness due to fracking wouldn't just be the
proximity to the well but the contaminants that may result from the
well. They'd like to measure the actual prevalence of contaminants
that may affect human health, but instead they assume that the prevalence
of contaminants is inversely correlated to the distance from the nearest
natural gas well.
\end{itemize}
\item \textbf{Discussion}: The authors list several possible explanations
for their findings. Why can\textquoteright t they narrow down their
results to yield a single, definite explanation?
\begin{itemize}
\item This is a sample survey, and sample surveys can't include causation.
All they can say is that there is a \textbf{\emph{correlation }}between
proximity to natural gas wells and certain health symptoms, but they
can't show what the cause of that correlation is. There may be a third
factor (exposure to contaminants) correlated to both variables, or
for all we know (though unlikely) people with health symptoms are
more likely to move in near natural gas wells.
\end{itemize}
\end{enumerate}
\end{xca}


\subsection{{[}point to in study{]} Random samples}
\begin{itemize}
\item The statistician's remedy to bias is to allow impersonal chance to
select the sample.
\begin{itemize}
\item A \textbf{simple random sample }gives all individuals an equal chance
of being selected.
\item A \textbf{spatially random sample }gives all \emph{geographic areas}
an equal chance of being selected.
\end{itemize}
\end{itemize}
\begin{defn}
A \textbf{simple random sample (SRS) }of size $n$ consists of $n$
individuals from the population chosen in such a way that every set
of $n$ individuals has an equal chance to be the sample actually
selected.

A \textbf{spatially random sample }of size $n$ consists of $n$ \emph{locations}
chosen in such a way that every set of $n$ locations has an equal
chance to be the sample actually selected. We then survey the people
living at those locations.
\end{defn}

\begin{example}
Drawing names from a hat: write {[}\# of students in class{]} names
on identical slips of paper and mix them in a hat. This is a population.
Now draw $10$ slips, one after the other. {[}do with webadvisor class
roster?{]} This is an SRS, because any $10$ slips have the same chance
as any other $10$.
\end{example}

\begin{xca}
There are (\#) students in this class; assume they're all sitting
in groups of $4$. I want to take a SRS of $4$ students in the class.
To do so, I write the numbers $1$ to (\#) on identical slips of paper.
I mix the slips in a hat and draw one at random. I then count clockwise
from my back-left in this group (point) and select the $n$th student
in that group. For example, if the number chosen is $3$, I choose
the third person clockwise from the back-left in the first group.
I continue this process in the second group, and so on for all the
groups. Every student has a $1$-in-$4$ chance of being selected
when I come to their group. Is this sample a SRS? Explain.
\begin{itemize}
\item To be an SRS, every group of $4$ students in the class must have
an equal chance of being selected.
\item In this case, groups of $4$ students will only have a maximum of
one student being selected, so seating groups have a $0\%$ chance
of being in the SRS. This is not an SRS.
\end{itemize}
\end{xca}

\begin{example}
How to use (pseudo)random numbers to choose an SRS {[}slide{]}. Joan\textquoteright s
small accounting firm serves 30 business clients. Joan wants to interview
a sample of 5 clients to find ways to improve client satisfaction.
To avoid bias, she chooses an SRS of size 5.
\begin{enumerate}
\item Label. Give each client a numerical label, using as few digits as
possible. Two digits are needed to label 30 clients, so we use labels
01, 02, 03,\dots ,28, 29, 30. {[}left of slide{]} It is also correct
to use labels 00 to 29 or even another choice of 30 two-digit labels.
Here is the list of clients, with labels attached, using 01 to 30:
{[}slide{]}
\item Table. Enter Table A anywhere and read two-digit groups. Suppose we
enter at line 130, which is 69051 64817 87174 09517 84534 06489 87201
97245. The first 10 two-digit groups in this line are 69 05 16 48
17 87 17 40 95 17. Each two-digit group in Table A is equally likely
to be any of the 100 possible groups, 00, 01, 02,\dots , 99. So two-digit
groups choose two-digit labels at random. That\textquoteright s just
what we want. 

Joan used only labels 01 to 30, so we ignore all other two-digit groups.
The first five labels between 01 and 30 that we encounter in the table
choose our sample. Of the first 10 labels in line 130, we ignore five
because they are too high (over 30). The others are 05, 16, 17, 17,
and 17. The clients labeled 05, 16, and 17 go into the sample. Ignore
the second and third 17s because that client is already in the sample.
Now run your finger across line 130 (and continue to line 131 if needed)
until five clients are chosen. The sample is the clients labeled 05,
16, 17, 20, 19. These are Bailey Trucking, JL Appliances, Johnson
Commodities, MagicTan, and Liu\textquoteright s Chinese Restaurant.
\end{enumerate}
\end{example}

\begin{itemize}
\item In practice, a lot of researchers use software to generate random
numbers. An example that is available on the Web is the Research Randomizer
at www.randomizer.org. Click on the link Randomize Now and fill in
the boxes. You can even ask the Randomizer to arrange your sample
in order. 
\item \includegraphics{\string"../../DATA 150 Data and Society/DATA 150 Class Notes/pasted2\string".png}
\end{itemize}
\begin{xca}
{[}slide{]} 2.2 Evaluating teaching assistants. To assess how its
teaching assistants are performing, the statistics department at a
large university randomly selects 3 of its teaching assistants each
week and sends a faculty member to visit their classes. The current
list of 20 teaching assistants is given below. Use the Research Randomizer
to choose 3 to be visited this week. Remember to begin by labeling
the teaching assistants from 01 to 20.
\end{xca}


\subsection{When can you trust a sample?}
\begin{itemize}
\item In order to be able to trust a sample, it has to be randomly selected.
Even so, there are some common downfalls:
\begin{itemize}
\item high \textbf{non-response}: only a small proportion of those randomly
selected for a survey respond. In this case, it is unclear whether
the respondents are \textbf{representative }of the entire population.
If not, the study suffers from \textbf{non-response bias}.
\item a \textbf{convenience sample}: individuals who are easily accessible
are more likely to be included in the sample. For example, stopping
people walking in the Bronx will not represent all of NYC.
\end{itemize}
\end{itemize}
\begin{xca}
If $50\%$ of Amazon reviews for a product are negative, do you think
this means that $50\%$ of buyers are dissatisfied with the product?
\end{xca}

\begin{example}
{[}slide{]} a Gallup poll about political identification. This poll
was randomly selected.
\end{example}


\section{{[}point to in study{]} Describing relationships between categorical
variables: contingency tables}
\begin{defn}
Recall that a variable is \textbf{numerical }if it can take a wide
range of numerical values and it is sensible to add, subtract, or
take averages with these values. 
\begin{example}
Federal spending per year (\textbf{continuous numerical variable:
}can take on fractional values.

Population (can only take on whole-number values: a \textbf{discrete
numerical variable})
\end{example}

A variable is \textbf{categorical }if its possbile values are categories.
The possible values of a categorical variable are called \textbf{levels.}
\begin{example}
The US state respondents live in has $51$ possible categories.

A survey asked respondents if they preferred no smoking ban, a partial
smoking ban, or a comprehensive smoking ban in their county. \textbf{Smoking\_ban
}is a categorical variable whose levels are ``none'', ``partial'',
and ``comprehensive''. Note that these categories have a natural
ordering even though they're not numbers. We call such a variable
an \textbf{ordinal variable}.
\end{example}

\end{defn}

\begin{itemize}
\item Whenever we analyze data, our principles will remain the same:
\begin{itemize}
\item First plot the data, then add numerical summaries.
\item Look for overall patterns and deviations from those patterns.
\item When the overall pattern is quite regular, there is sometimes a way
to describe it very briefly.
\end{itemize}
\item Today we'll read the results of a study which examines the relationship
between distance from natural gas wells and the prevalence of various
health conditions.
\end{itemize}
\begin{xca}
Look at Table $1$ from the study. Name an explanatory variable and
at least five response variables. What is the target population in
the study? The sample(s)? Are each of the variables categorical or
numerical? If categorical, are they ordinal or non-ordinal? If numerical,
are they continuous or discrete?
\begin{itemize}
\item The categorical explanatory variable is ``proximity to the nearest
natural gas well'' and has three categories: $<1$ km, $1$-$2$
km, and $>2$ km. This is an ordinal, categorical variable because
it's possible to order the levels, but it doesn't make sense to average,
for example, ``$>2$ km'' and ``$1$-$2$ km'' because it's unclear
how to compute the average.
\begin{itemize}
\item This explanatory variable is used for two samples: households and
individuals.
\end{itemize}
\item There are multiple categorical response variables:
\begin{itemize}
\item sex (categorical non-ordinal)
\item whether the respondent has children (categorical non-ordinal)
\item the occupation of the respondent (categorical non-ordinal)
\item whether they drank ground-fed water (categorical non-ordinal)
\item whether their water has an unnatural appearance (categorical non-ordinal)
\item ...
\end{itemize}
\item There are multiple numerical response variables:
\begin{itemize}
\item education (years) (continuous numerical)
\item age (years) (continuous numerical)
\item years in household (continuous numerical)
\item body mass index (BMI) (continuous numerical)
\item ...
\end{itemize}
\end{itemize}
\end{xca}

\begin{itemize}
\item Because this study has so many variables, we'll isolate just a few
to think about. In particular, we'll focus on the categorical variables
\emph{occupation }and \emph{presence of health issues}.
\item \textbf{Goal: }examine whether greater proximity to natural gas wells
leads to a different distribution of job fields and/or health issues.
\begin{itemize}
\item We can display the relationship between two categorical variables
in a \emph{contingency table }or \emph{two-way table}.
\end{itemize}
\item {[}point to in study p24{]} A \textbf{contingency table }or \textbf{two-way
table }displays the relationship between two categorical variables.
\begin{itemize}
\item The \textbf{row variable} in a contingency table is the variable where
each row of the table describes one of the possible values of the
variable. 
\end{itemize}
\item To best grasp a contingency table, look at the \textbf{distribution
}of each variable (row/column) separately. The \textbf{distribution}
of a categorical variable is how often each outcome (for a sample
survey, each possible response) occurred.
\item To describe relationships among categorical variables, it's useful
to \textbf{calculate the percentage of each row that falls in each
column and vice versa.}
\end{itemize}
\begin{xca}
{[}slide{]} The faculty at the University of Utah (unfortunately,
I can't find the corresponding statistics for Westminster) fall into
the following distributions:

\includegraphics[scale=0.75]{pasted4}
\begin{enumerate}
\item Invent a research question based on this table.
\item What are the row and column variables in the table?
\item Convert the relevant row(s) of the table to percentages in order to
better answer your research question.
\item How would you test your hypothesis? What additional information would
you need in order to do so? Find this information and explain how
you would answer your research question given this information.
\begin{enumerate}
\item You'd need the demographics of faculty nationwide or of the US or
Utah to compare against.
\item If the percentages were substantially different from those in the
general population, one might conclude that specific groups are over-/underrepresented
in the U faculty.
\end{enumerate}
\end{enumerate}
\end{xca}

\begin{itemize}
\item You may have examined population demographics and compared to the
U faculty demographics to see how big the difference was and how much
specific groups were over-/underrepresented in the U's faculty. This
is pretty much what statisticians do when they analyze for over/underrepresentation,
except they compute \emph{test statistics} and give ranges or \emph{confidence
intrervals} for the proportion of the population in each category
and use numbers to draw a conclusion.
\item We won't overly focus on the numerical process, but know that it's
called a \emph{chi-square test} and can be performed fairly easily
by computer.
\end{itemize}
\begin{question}
If we see a substantial difference in the percentage of faculty that
are women vs. the percentage of the general population, is this necsesarily
hard evidence of discrimination?
\end{question}

\begin{itemize}
\item In order to measure the extent of a correlation between gender and
admission decision, or between proximity to natural gas wells and
the presence of health issues, it's necessary to perform \textbf{statistical
inference}.
\begin{itemize}
\item You won't be expected to perform inference yourself, just understand
and interpret other people's inference.
\end{itemize}
\end{itemize}

\section{Statistical inference}
\begin{itemize}
\item To \textbf{infer }is to draw a conclusion from evidence.
\item \textbf{Statistical inference }draws a conclusion about a population
from evidence provided by a sample.
\begin{itemize}
\item Statistical conclusions are uncertain, because the sample isn't the
entire population. 
\end{itemize}
\item Statistical inference has to not only state conclusions, but also
say how uncertain they are.
\begin{itemize}
\item We use the language of probability to express uncertainty.
\item Hey, that's convenient that we just learned some probability!
\end{itemize}
\item Texts and courses intended to train people to \emph{do }statistics
spend most of their time on inference.
\item Our aim is to help you \emph{understand} statistics, so we'll look
only at a few basic techniques of inference.
\item To start, think about what you already know, and don't be too impressed
by elaborate statistical techniques.
\begin{itemize}
\item Even the fanciest inference can't remedy basic flaws such as voluntary
response samples or uncontrolled experiments.
\end{itemize}
\end{itemize}
\begin{example}
A pickup basketball player at the gym says he makes $80\%$ of his
free throws. ``Show me,'' you say. He shoots $20$ free throws and
makes $8$ of them. ``Aha,'' you conclude. ``If he makes $80\%$,
he would almost never make as few as $8$ of $20$. So I don't believe
his claim.'' That's the reasoning of statistical \textbf{tests of
significance}: \emph{an outcome that is very unlikely if a claim is
true is good evidence that the claim is not true}.
\end{example}

\begin{defn}
\textbf{Statistical inference} uses data from a sample to draw conclusions
about a population. Statistical tests ask if sample data give good
evidence \emph{against} a claim. A statistical test says, ``If we
took many samples and the claim were true, we would rarely get a result
like this.''
\end{defn}


\subsection{Hypotheses and $p$-values}
\begin{itemize}
\item In most studies, we hope to show that some definite effect is present
in the population.
\item A statistical test begins by supposing for the sake of argument that
the effect we seek is \emph{not }present.
\item We then look for evidence against this supposition and in favor of
the effect we hope to find.
\item The first step in a test of significance is to state a claim that
we will try to find evidence \emph{against}.
\end{itemize}
\begin{defn}
The claim being tested in a statistical test is called the \textbf{null
hypothesis}, abbreviated $H_{0}$. Here ``null'' means ``nothing'',
in the sense that ``nothing's happening'' or ``no effect is present''.
The test is designed to assess the strength of the evidence against
the null hypothesis.

The statement we hope or suspect is true instead of $H_{0}$ is called
the \textbf{alternative hypothesis }and is abbreviated $H_{0}$. 
\end{defn}

\begin{example}
For example, in the basketball example above, the alternative hypothesis
offered by the player is that the proportion $\pi$ of free throws
he makes is $80\%$s; that is, $H_{A}:\pi=0.8$. We call $\pi$ the
\textbf{population proportion}; in this case, the population proportion
is the proportion of all free throws shot, say, in the last year that
the player makes.

The skeptical (but unbiased) response is that the player is just as
likely to miss a free throw as to make one; that is, $H_{0}:\pi=0.5$.
\end{example}

\begin{xca}
{[}Four Studies slide{]} For each of the following studies, do each
of the following: 
\begin{itemize}
\item Describe all variables being measured in the study. Identify whether
these variables are numerical or categorical. If they are numerical,
determine whether they are discrete or continuous. If they are categorical,
determine if they are ordinal or non-ordinal. 
\item Identify a null and alternative hypothesis for the study. 
\item Is the study an observational study or an experiment? Is the study
able to show that one variable~caused a change in another? If so,
do you believe that such a conclusion was justified? 
\item What possible lurking variables might be relevant to consider?
\end{itemize}
\begin{enumerate}
\item To determine whether coffee drinkers can tell the difference between
fresh-brewed and instant coffee, an experiment is conducted in which
each of $50$ subjects tastes two unmarked cups of coffee and says
which they prefer. We find that $36$ out of our $5$0 subjects choose
the fresh coffee, a rate of about $72\%$.
\item In the Detroit Area Study, which aims to test attitudes toward neighborhood
racial segregation, random samples of $2647$ adults in the Detroit
metro area are asked if they would move away from their neighborhood
if Black folks moved into $3$ out of every $15$ houses. (This is
the actual population of Black folks in the Detroit metro area.) In
$1976$, $24\%$ of whites said they would try to leave. By $1992$,
this percentage had dropped to $15\%$.
\item A June $2018$ Pew Research Center survey was conducted among $2,002$
adults. It showed that $73\%$ of Americans support granting permanent
legal status to undocumented immigrants who came to the US when they
were children, as opposed to $20\%$ which disapprove.
\item One study in the Washington, D.C., area shows that a bank rejected
$17.5\%$ of home mortgage loan applicants who were Black but only
$3.3\%$ of applicants who were white.
\begin{enumerate}
\item Lurking variables: lower income, poorer credit records, less secure
jobs (correlated with being a PoC)
\end{enumerate}
\end{enumerate}
\end{xca}

\begin{itemize}
\item How does a statistical study draw conclusions about which hypothesis
is true?
\item A statistical study is conducted entirely under the assumption that
the null hypothesis is true. The study then presents evidence in the
form of a \textbf{test statistic}, a number that summarizes the responses
of the sample. The study calculates a \textbf{$p$-value}, the probability
that this test statistic would be observed if the null hypothesis
were true. Traditionally, if $p<5\%$ or $.05$, the null hypothesis
is \textbf{rejected} as too unlikely in the face of the evidence.
\end{itemize}
\begin{example}
Commonly, test statistics take the form of proportions, percentages,
averages, or even more complex formulas.

In the basketball example above, our test statistic is the \textbf{sample
proportion }$\hat{p}$ of the $20$ observed free throws that were
made. So $\hat{p}=\frac{8}{20}=0.4$. The observer notes that the
test statistic $\hat{p}=0.4$ is fairly consistent with the null hypothesis
$H_{0}:p=0.5$. Hence, we \textbf{fail to reject the null hypothesis
}in favor of the player's claim. The null hypothesis that the player
is only a $50\%$ free throw shooter stands.
\end{example}

\begin{xca}
{[}Four Studies slide{]} Describe all test statistics in the above
four studies. Based on each test statistic, do you think you'd reject
your null hypothesis as being too unlikely? How could we quantify
the unlikeliness of a null hypothesis?
\end{xca}

\begin{defn}
The probability, computed assuming $H_{0}$ is true, that the sample
outcome would be as extreme or more extreme than the actually observed
outcome is called the \textbf{p-value} of the test. The smaller the
p-value is, the stronger the evidence against $H_{0}$ provided by
the data.
\end{defn}

\begin{itemize}
\item In practice, most statistical tests are carried out by computer software
that calculates the P-value for us. 
\item It is usual to report the p-value in describing the results of studies
in many fields. 
\item You should therefore understand what p-values say even if you don\textquoteright t
do statistical tests yourself, just as you should understand what
\textquotedblleft 95\% confidence\textquotedblright{} means even if
you don\textquoteright t calculate your own confidence intervals.
\item Time to analyze the fracking data!
\end{itemize}
\begin{xca}
EJ Worksheet $1$
\end{xca}


\subsection{Reading Question: Residential Proximity to Major Highways}
\begin{enumerate}
\item Introduction: Explain, citing evidence from the article, how proximity
to major traffic roads is related to public health. 
\begin{enumerate}
\item Exposure to motor vehicle pollutants known to cause health problems
\item associated with asthma, COPD, and other respiratory symptoms
\end{enumerate}
\item Introduction: This study looked at all people who lived within 150
meters of a major roadway. Why did the study choose this number (150)
as the cutoff? 
\begin{enumerate}
\item traffic-related pollutants diminish to near-background levels after
$150$ m 
\end{enumerate}
\item Introduction: What do the authors hypothesize the relationship between
socioeconomic status and air pollution exposure is? 
\begin{enumerate}
\item they hypothesize due to ``widely accepted'' evidence that economically
poorer people are more likely to be exposed to air pollutants
\end{enumerate}
\item Results: Which variables that were looked at saw large disparities
of percentages living within 150 meters of a major highway? Which
variables saw small/no disparities? 
\begin{enumerate}
\item large disparities: race/ethnicity, nativity, and language spoken at
home
\item small/no disparities: sex, age, region, education, poverty status
\end{enumerate}
\item Discussion: What is the correlation between living in urban areas
and people exposure to a major highway? Explain what this correlation
value means.
\begin{enumerate}
\item $R=0.65$
\item $R^{2}=0.4225$
\item So $42.25\%$ of the variation in highway exposure is explained by
urbanity
\end{enumerate}
\item Discussion: In light of your answer for (4), what is a possible lurking
variable in this study? Explain how this new variable could affect
the results.
\begin{enumerate}
\item Exposure to multiple major roadways
\end{enumerate}
\end{enumerate}
\begin{itemize}
\item A more recent article on proximity to highways and cognitive impacts
in China: https://www.sciencedirect.com/science/article/abs/pii/S0048969720361362
\end{itemize}

\subsection{Warmup: thinking about confidence intervals}
\begin{note}
The article states, ``living in a household < 1 km from the nearest
gas well remained associated with increased reporting of skin conditions
{[}odds ratio (OR) = 4.13; 95\% confidence interval (CI): 1.38, 12.3{]}
and upper respiratory symptoms (OR = 3.10; 95\% CI: 1.45, 6.65) compared
with households > 2 km from the nearest gas well. What do they mean
by ``95\% confidence interval'', and how do they compute it?
\end{note}

\begin{itemize}
\item A \textbf{level C confidence interval} for a parameter has two parts:
\begin{itemize}
\item An \textbf{interval }calculated from the data.
\item A \textbf{confidence level }C, which gives the probability that the
interval will capture the true parameter value in repeated samples.
\item Then the level $C$ confidence interval for this parameter is 
\[
\text{estimate}\pm ME_{C}
\]
\item {[}highlight Trump approval poll and study{]} every poll has error.
So polls often report a \textbf{confidence interval }and/or a \textbf{margin
of error: }the range of values in which they are $95\%$ certain that
the true value falls.
\item {[}site: https://thehill.com/hilltv/what-americas-thinking/445989-hill-harrisx-poll-trump-job-approval-at-44-percent-down{]}
So, in this poll, we are $95\%$ sure that the true Trump approval
rating in May was between 
\[
(44-3.1,44+3.1)=(40.9\%,47.1\%).
\]
\end{itemize}
\end{itemize}

\subsection{Warmup: thinking about confidence intervals}
\begin{xca}
(TPS) Suppose that a political poll found with 95\% confidence that
between 40.3\% and 43.2\% of voters support a given candidate. Which
of the following interpretations of this confidence statement is accurate?
Explain why or why not, relating each statement to the archery analogy
when appropriate. (Take your time and digest each statement; some
of these are tricky!) 
\begin{enumerate}
\item If the pollsters take 100 polls with different samples, 95 of the
resulting sample proportions will fall in this interval. 

False. The circle will contain the bull's-eye 95\% of the time, not
the other arrows.
\item If the pollsters take 100 polls with different samples, we'd expect
roughly 95 of the resulting confidence intervals will contain the
true population proportion. 

True. We'd expect roughly 95\% of the circles will contain the bull's-eye.
\item If the pollsters take 100 polls with different samples, we'd expect
exactly 95 of the resulting confidence intervals to contain the true
population proportion. 

False, for the same reason that saying exactly $500$ out of $1000$
coin flips will be heads is false.
\item We can be certain that less than half of voters support the candidate. 

False; we can't be certain of anything here. These are probability
statements.
\item We can be 95\% sure that the candidate would lose the election if
it were held tomorrow.

False; the confidence interval only tells us about random sampling
error, not any other form of error. 
\item In the long run, over many trials, 95 out of 100 population proportions
will fall in the interval.

There's only one bull's-eye; the bull's-eye is not moving around,
our interval is.
\end{enumerate}
\end{xca}

\begin{itemize}
\item Out of $20$ political polls, one of their reported estimates and
ME won't capture the true population proportion!
\end{itemize}
\begin{xca}
Whole-class discussion: EJ WS 1 \#9-14
\end{xca}

\begin{itemize}
\item Check-in: how are folks doing with EJ WS 2? Have students present
to class.
\item What EJ questions have these WSs not gotten at that you want to discuss?
Investigate online and find statistical studies. Determine whether
you trust them.
\item {[}if time{]} start climate change
\end{itemize}
\begin{xca}
\textbf{{[}Worksheets{]} }EJ Worksheets $2$
\begin{itemize}
\item ''The environmental justice literature suggests that socially disadvantaged
groups might experience a phenomenon known as \textquotedbl triple
jeopardy\textquotedbl{} (37). 
\begin{itemize}
\item First, poor and minority groups are known to suffer negative health
effects from social and behavioral determinants of health (e.g., psychosocial
stress, poor nutrition, and inadequate access to health care). 
\item Second, as suggested in this analysis, certain populations (e.g.,
members of minoritized communities, foreign-born persons, and persons
who speak a non-English language at home) might be at higher risk
for exposure to traffic-related air pollution and fracking-related
water and air pollution as a result of residential proximity to major
highways. 
\item Third, there is evidence suggesting a multiplicative interaction between
the first two factors, such that socially disadvantaged groups experience
disproportionately larger adverse health effects from exposure to
air pollution (37\textendash 39).''
\end{itemize}
\end{itemize}
\end{xca}


\end{document}
