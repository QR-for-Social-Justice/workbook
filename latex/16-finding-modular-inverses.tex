%% LyX 2.3.6.2 created this file.  For more info, see http://www.lyx.org/.
%% Do not edit unless you really know what you are doing.
\documentclass[oneside,english]{amsart}
\usepackage[T1]{fontenc}
\usepackage{geometry}
\geometry{verbose,tmargin=1in,bmargin=1in,lmargin=1in,rmargin=1in}
\usepackage{amstext}
\usepackage{amsthm}
\usepackage{graphicx}

\makeatletter
%%%%%%%%%%%%%%%%%%%%%%%%%%%%%% Textclass specific LaTeX commands.
\numberwithin{equation}{section}
\numberwithin{figure}{section}
\theoremstyle{plain}
\newtheorem{thm}{\protect\theoremname}
\theoremstyle{definition}
\newtheorem{xca}[thm]{\protect\exercisename}
\theoremstyle{definition}
\newtheorem{example}[thm]{\protect\examplename}

\makeatother

\usepackage{babel}
\providecommand{\examplename}{Example}
\providecommand{\exercisename}{Exercise}
\providecommand{\theoremname}{Theorem}

\begin{document}
\title{WCSAM 206 - Finding Multiplicative Inverses Modulo $n$}
\title{}
\maketitle

\section{Last Time}
\begin{itemize}
\item We used the Euclidean Algorithm to find the $\gcd$ of pairs of large
numbers.
\item If $\gcd(m,n)=1$, then the affine cipher $C\equiv mP+k\mod n$ works
to encrypt information (for any additive key $k$).
\item If $\gcd(m,n)\neq1$, then the affine cipher doesn't work.
\end{itemize}
\begin{xca}
\textbf{(Preview Activity) }Using the Euclidean Algorithm we learned
in class, give each of the following questions your best attempt,
showing all work. Then bring your work to class on the date below
for discussion.
\end{xca}

\begin{enumerate}
\item Compute the following gcds: 
\begin{enumerate}
\item $\gcd\left(87,1364\right)$
\begin{enumerate}
\item We compute
\begin{align*}
1364 & =87(15)+59\\
87 & =59(1)+28\\
59 & =28(2)+3\\
28 & =3(9)+1\\
3 & =1(3)+0
\end{align*}
so that $\gcd(87,1364)=1$.
\end{enumerate}
\item $\gcd\left(103011,111411\right)$
\begin{enumerate}
\item We compute
\begin{align*}
111411 & =103011(1)+8400\\
103011 & =8400(12)+2211\\
8400 & =2211(3)+1767\\
2211 & =1767(1)+444\\
1767 & =444(3)+435\\
444 & =435(1)+9\\
435 & =9(48)+3\\
9 & =3(3)+0
\end{align*}
hence $\gcd(103011,1114111)=\gcd(3,0)=3$.
\item You could have determined that $103011$ and $111411$ share a factor
of $3$ by adding their digits up and verifying that the result is
divisible by $3$.
\end{enumerate}
\end{enumerate}
\item Which of the affine ciphers below are decryptable? Explain your answer. 
\begin{enumerate}
\item $C\equiv255P+251\mod256$ (an ASCII cipher) 
\begin{enumerate}
\item We compute
\begin{align*}
256 & =255(1)+1\\
255 & =1(255)+0
\end{align*}
so that $\gcd(255,256)=1$ and the cipher is decryptable.
\end{enumerate}
\item $C\equiv2228225P+5738254\mod1114112$
\begin{enumerate}
\item We compute
\begin{align*}
2228225 & =1114112(2)+1\\
1114112 & =1(1114112)+0
\end{align*}
so that $\gcd(2228225,1114112)=1$ and the cipher is decryptable.
\end{enumerate}
\item $C\equiv-4P+42\mod532$
\end{enumerate}
\end{enumerate}
\begin{example}
Does the affine cipher $C\equiv343P+3\mod454$ work to transmit a
message? Use the Euclidean Algorithm to answer this question.
\begin{itemize}
\item This time, we'll write out our steps very carefully in the Euclidean
Algorithm, because we'll be coming back and doing each step backwards
in order to find $343^{-1}\mod454$.
\end{itemize}
\begin{enumerate}
\item Since $454>343$, we reduce $454\mod343$. This is the same as dividing
$454$ by $343$ and keeping the remainder:
\begin{equation}
454=343(1)+111\label{eq:step-1}
\end{equation}
so $\gcd(343,454)=\gcd(343,111)$.
\item Now, since $343>111$, we divide $343$ by $111$ and keep the remainder:
\begin{equation}
343=111(3)+10\label{eq:step-2}
\end{equation}
so that $\gcd(343,111)=\gcd(111,10)$.
\item Since $111>10$, we divide again:
\begin{equation}
111=10(11)+1\label{eq:step-3}
\end{equation}
so that $\gcd(111,10)=\gcd(10,1)=1$.
\end{enumerate}
\end{example}

\begin{itemize}
\item Same old, same old. But hidden within the Euclidean Algorithm is a
method for finding multiplicative inverses $\mod n$! 
\item We can use our work above to find $343^{-1}\mod454$.
\item The key is to use the EA to find $x$ and $y$ so that $343x+454y=1$.
If we can do this, then we can reduce both sides modulo $454$ to
get
\[
343x\equiv1\mod454
\]
and $x$ will be the multiplicative inverse of $343$!
\end{itemize}
\begin{example}
Look closer at (\ref{eq:step-3}): there's a 1 there! If we could
isolate that 1 and get it equal to some multiple of 343 plus some
multiple of 454, we'd be done!
\end{example}

Well, let's play around with it:
\[
1=111-10(11)
\]

Darn, $111$ and $10(11)$ aren't multiples of $343$ or $454$! But
from (\ref{eq:step-2}), $10=343-111(3)$, so
\begin{eqnarray*}
1 & = & 111-[343-111(3)](11)\\
 & = & 111-343(11)+111(33)\\
 & = & 111(34)-343(11).
\end{eqnarray*}
Now all we need is to make $111$ look like some multiples of $343$
added/subtracted from some multiples of $454$! Well, via (\ref{eq:step-1}),
\[
111=454-343
\]
so that
\begin{eqnarray*}
1 & = & 454(34)-343(34)-343(11)\\
1 & = & 454(34)-343(45)
\end{eqnarray*}
hence $x=-45$ and $y=34$! 
\begin{itemize}
\item This means that $343(-45)+454(34)=1$, and hence that
\[
343(-45)\equiv1\mod454.
\]
Hence, the multiplicative inverse of $343$ is $343^{-1}=-45\equiv409\mod454$.
\item Therefore, if we want to decrypt a message that has been enciphered
using the affine cipher $C\equiv343P+3\mod454$, we can get a decryption
equation by first subtracting $3$ from both sides and then multiplying
by $-45$:
\begin{align*}
C-3 & \equiv343P\mod454\\
-45(C-3) & \equiv P\mod454.
\end{align*}
\end{itemize}
\rule[0.5ex]{1\columnwidth}{1pt}

{[}start here 10-16-18{]}
\begin{itemize}
\item Recap: say we want to find the multiplicative inverse of $a\mod n$.
Then, if we:
\begin{enumerate}
\item Perform the Euclidean Algorithm on $a$ and $n$ until you get $1$
as one of the numbers on the left side of a sheet of paper.
\item Solve each of the resulting equations for their remainders on the
right side of each of the equations.
\item Starting with the bottom remainder equation which equals $1$, plug
in each preceding remainder equation in place of the remainders, collecting
like terms together between plugging in. Be careful not to explicitly
multiply the remainders or the original two numbers.
\item Reduce the resulting equation modulo $n$ to see what $a^{-1}\mod n$
is.
\item Check your answer by multiplying $a$ by your conjectured inverse
and reducing $\mod n$ to see if you get $1$.
\end{enumerate}
\item Why don't you try another example:
\end{itemize}
\begin{xca}
Find the multiplicative inverse of $15\mod28$. Then use this inverse
to decrypt the message 
\[
\text{!C! GWE?KTA}
\]
given that it was encrypted using the affine cipher $C\equiv15P+2\mod28$
and the alphabet ABCDEFGHIJKLMNOPQRSTUVWXYZ!?

\textbf{Note}: this problem would take forever if we used ``guess
and check''.
\begin{enumerate}
\item Apply the Euclidean Algorithm to find $\gcd(15,28)$ until we get
a remainder of $1$ (if we can't get a remainder of $1$, there is
no multiplicative inverse):
\begin{align*}
28 & =15(1)+13\\
15 & =13(1)+2\\
13 & =2(6)+1
\end{align*}
\item Solve each of the resulting equations for their remainders:
\begin{align*}
13 & =28-15(1)\\
2 & =15-13(1)\\
1 & =13-2(6)
\end{align*}
\item Starting with the bottom equation which equals $1$, plug in each
of the preceding equations in place of the remainders and collect
like terms after each step:
\begin{align*}
1 & =13-2(6)\\
 & =13-[15-13](6)\\
 & =13-15(6)+13(6)\\
 & =13(7)-15(6)\\
 & =[28-15](7)-15(6)\\
 & =28(7)-15(7)-15(6)\\
1 & =28(7)-15(13)
\end{align*}
\item Reduce the resulting equation mod $n$:
\[
1\equiv0\cdot7-15(13)\equiv15(-13)\mod28
\]
so that $15^{-1}=-13\equiv15\mod28$. $15$ is its own inverse mod
$28$, but there's no way we would have just guessed that without
somehow remembering that $15^{2}=224+1$ and that $224$ is a multiple
of $28$.
\item Check that your inverse actually gives you $1$ when multiplied by
$15$:
\[
15\times15=225=28(8)+1\equiv1\mod28
\]
\item Now, we use the fact that $15^{-1}=15\mod28$ to solve $C\equiv15P+2\mod28$
for $P$:
\begin{align*}
C-2 & \equiv15P\mod28\\
15(C-2) & \equiv P\mod28.
\end{align*}
\item Plugging into this equation, we get that the message is ``yay euclid''
:)
\end{enumerate}
\end{xca}

~
\begin{xca}
A message is encrypted using the equation $C\equiv97P+31\mod233$.
Find the decryption equation for this affine cipher.
\begin{enumerate}
\item Use the Euclidean Algorithm to show that $\gcd(97,233)=1$ and hence
that this cipher works.
\item Use the Extended Euclidean Algorithm to find $x$ and $y$ so that
$97x+233y=1$:

\includegraphics[scale=0.5]{\string"Old Cryptography Notes/pasted53\string".png}
\item Use the inverse of $97\mod233$ to solve the encryption equation for
$P$:
\begin{align*}
C-31 & \equiv97P\mod233\\
-12(C-31) & \equiv P\mod233.
\end{align*}
\end{enumerate}
\end{xca}

~
\begin{xca}
Use the extended Euclidean Algorithm to find the decyption equation
corresponding to each of the following encryption equations, or say
why the cipher doesn't work:
\begin{enumerate}
\item $C\equiv24P+7\mod54$
\item $C\equiv19P+3\mod26$
\item $C\equiv57P+43\mod81$
\item $C\equiv15P+8\mod38$
\item $C\equiv29P-4\mod40$
\item $C\equiv8P\mod49$
\end{enumerate}
\end{xca}

\includegraphics{\string"Old Cryptography Notes/pasted130\string".png}
\end{document}
