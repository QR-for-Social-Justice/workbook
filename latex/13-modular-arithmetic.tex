%% LyX 2.3.6.2 created this file.  For more info, see http://www.lyx.org/.
%% Do not edit unless you really know what you are doing.
\documentclass[oneside,english]{amsart}
\usepackage[T1]{fontenc}
\usepackage{geometry}
\geometry{verbose,tmargin=1in,bmargin=1in,lmargin=1in,rmargin=1in}
\setcounter{secnumdepth}{2}
\setcounter{tocdepth}{2}
\usepackage{amstext}
\usepackage{amsthm}
\usepackage{graphicx}

\makeatletter

%%%%%%%%%%%%%%%%%%%%%%%%%%%%%% LyX specific LaTeX commands.
%% Because html converters don't know tabularnewline
\providecommand{\tabularnewline}{\\}

%%%%%%%%%%%%%%%%%%%%%%%%%%%%%% Textclass specific LaTeX commands.
\numberwithin{equation}{section}
\numberwithin{figure}{section}
\theoremstyle{plain}
\newtheorem{thm}{\protect\theoremname}
\theoremstyle{definition}
\newtheorem{xca}[thm]{\protect\exercisename}
\theoremstyle{definition}
\newtheorem{example}[thm]{\protect\examplename}
\theoremstyle{definition}
\newtheorem*{defn*}{\protect\definitionname}
\theoremstyle{definition}
\newtheorem*{example*}{\protect\examplename}
\theoremstyle{definition}
\newtheorem*{xca*}{\protect\exercisename}

\makeatother

\usepackage{babel}
\providecommand{\definitionname}{Definition}
\providecommand{\examplename}{Example}
\providecommand{\exercisename}{Exercise}
\providecommand{\theoremname}{Theorem}

\begin{document}
\title{WCSAM 206 Day 3 - Introduction to the Affine Cipher \& Modular Arithmetic}
\maketitle

\section{Growth Mindset video response}
\begin{xca}
(RQ) TPS: Reading questions: views of intelligence//Read this article
(Links to an external site.) from the American Psychological Association
(APA) on how \textquotedbl Believing You Can Get Smarter Makes You
Smarter\textquotedbl{} and post in the thread below short (2-3 sentence)
answers to the following questions:
\begin{enumerate}
\item How would you describe a \textquotedbl growth mindset\textquotedbl{}
in your own words? 
\item Name one skill you developed or one thing you learned to do through
hard work and making mistakes. Do you believe that it's possible to
learn math through hard work and making mistakes, or is math somehow
different from music and basketball in this way? Why or why not? 
\item What can you do in this course to promote a growth mindset about your
own abilities? 
\begin{enumerate}
\item Most students mention that understanding takes time and work to affirm
one's own knowledge
\end{enumerate}
\item What can your professor and fellow students do in this course to promote
a growth mindset about your own abilities? 
\begin{enumerate}
\item Discuss my own struggles with math
\item Everyone can learn math; mathematical ability is not fixed and can
change with work. 
\item I deeply hope that this course will not be like the typical math courses
you've taken in the past in the sense that you won't be forced to
do a bunch of computations in your head. If you're able to make little
creative leaps based on what we've discussed in class, you'll do wonderfully
in this course.
\item In this course, you won't be graded on how quickly you learn material.
Corrections {[}mention including exams{]}, weighting of your final
grade toward the final exam/just coming to class having attempted
problems, I hope will promote that mindset.
\end{enumerate}
\end{enumerate}
\end{xca}

\begin{itemize}
\item A statement from the American Mathematical Society, hosts of the largest
annual mathematics meeting in the world and organization that strives
to represent research mathematicians worldwide:
\begin{itemize}
\item The AMS strives to ensure that participants in its activities enjoy
a welcoming environment. In all its activities, the AMS seeks to foster
an atmosphere that encourages the free expression and exchange of
ideas. The AMS supports equality of opportunity and treatment for
all participants, regardless of gender, gender identity or expression,
race, color, national or ethnic origin, religion or religious belief,
age, marital status, sexual orientation, disabilities, veteran status,
or immigration status.
\end{itemize}
\item I would add socioeconomic status and mental illness to this list for
my course policy (the latter could be considered covered under ``disability'',
but I want to make this explicit).
\item Title IX is a federal law that states: ``No person in the United
States shall, on the basis of sex, be excluded from participation
in, be denied the benefits of, or be subjected to discrimination under
any education program or activity receiving Federal financial assistance.''
I'd expand policy to include any of the categories mentioned above.
\item Affirm that all students can excel in the course. Making this real
by helping students understand their pathways to excelling. Undermining
the potential interpretation of courses as filters for past experiences
or weed-out contexts. Perhaps normalizing your own past struggles
along the path to success.
\item Resources that may help you succeed in this and other classes on campus:
\begin{itemize}
\item food pantry in Carleson Hall
\item free counseling services at the Counseling Center, which provides
``Brief counseling and referral services for individuals who may
be experiencing psychological or emotional difficulties''.
\item we'll talk multiple times throughout the semester about developing
study skills that will benefit you during your college experience.
{[}story of failing first exam; my goal to make sure you're prepared
with study tips/skills early and often. Usually everyone passes my
first midterm partially because of these skills!{]}
\end{itemize}
\end{itemize}

\section{Why modular arithmetic?}

\subsection{Decrypting shift ciphers with known key}
\begin{xca}
\textbf{(RQ1) }Try to apply al-Kindi's technique to the ciphertext
\textquotedbl bpiwtbpixrpxhiwtqthiegdvgpb\textquotedbl . Assume
that you know this message was encrypted with a simple shift cipher,
but you don't know what the key (shift) is. Bring your work to class,
on paper, on the day below for class discussion.
\end{xca}

\begin{itemize}
\item Here's a quicker way to decrypt shift ciphers.
\item Let's say we're trying to decrypt ``FZIYMDXFGVHVMWZNOMVKKZMVGDQZ''
(``Mathematica is the best program'') with $k=21$. That means that
we're going to shift every letter \textbf{back} by 21 letters.
\item Try shifting ``F'' back by 21 letters. 
\item Now we have to do it for the rest of the code. It's going to take
a while, right? Any ideas on how to speed this up?
\item Well, there are 26 letters in the alphabet. Shifting ``F'' back
by 21 gives us: 
\[
f,e,d,c,b,a,z,y,x,w,v,u,t,s,r,q,p,o,n,m,l,\underline{k}.
\]
\item Notice this is the same result as you get when shifting ``F'' \textbf{forward}
by 5. The alphabet is cyclical, so shifting back by 15 is the same
as shifting forward by 11.
\item Now let's shift ``Z'' forward by 5. 
\item This is where it's beneficial to have a \textbf{letter-number correspondence
sheet}; number the letters in increasing order starting with $a=0$.
{[}see the last page of your updated CoursePack.{]}

\includegraphics{\string"Old Cryptography Notes/pasted42\string".png}
\item Well, ``Z'' is the 25th letter in the alphabet. Adding 5 to 25 gives
$30$. There's no 30th letter of the above alphabet, right? But the
25th letter is ``z'', so the ``26th letter'' should be a shift
of one letter from ``z''. That gives ``a''! \emph{What we're really
saying here is that, in the sense of the alphabet, $z+1=25+1=0=a$,
}just like $23:00+1=0:00$ in military time!
\item Now $z+5=25+5=30$ is $4$ more than a, which is $e=4$!
\item Keep going\textendash it's straightforward for p and i. For w, we
get 33, which is $26+7$. So it's a shift of 7 from ``z'', which
is the same as the 7th letter of the alphabet, which is \textbf{h}. 
\end{itemize}
\begin{tabular}{|c|c|c|c|c|c|c|c|c|c|c|c|c|c|c|c|c|c|c|c|c|c|c|c|c|c|c|c|}
\hline 
Ciphertext & b & p & i & w & t & b & p & i & x & r & p & x & h & i & w & t & q & t & h & i & e & g & d & v & g & p & b\tabularnewline
\hline 
\hline 
Ciphertext number & 1 & 15 & 8 & 22 & 19 & 1 & 15 & 8 & 23 & 17 & 15 & 23 & 7 & 8 & 22 & 19 & 16 & 19 & 7 & 8 & 4 & 6 & 3 & 21 & 6 & 15 & 1\tabularnewline
\hline 
Plaintext number & 12 & 1 & 19 & 33 & 4 &  &  &  &  &  &  &  &  &  &  &  &  &  &  &  &  &  &  &  &  &  & \tabularnewline
\hline 
Plaintext & \textbf{m} & \textbf{a} & \textbf{t} & \textbf{h} & \textbf{e} & ...you finish this. &  &  &  &  &  &  &  &  &  &  &  &  &  &  &  &  &  &  &  &  & \tabularnewline
\hline 
\end{tabular}
\begin{example}
Note: somehow $33=7$ in the alphabet! This is like thinking of the
alphabet as a clock {[}draw{]} with $26$ numbers on it, going from
$0$ on the top to $25$. Where is 33:00 on this clock? It's aka 7:00!
\end{example}


\section{Introduction to Modular Arithmetic}
\begin{itemize}
\item \textbf{Today}: we'll introduce prime factorizations and the Euclidean
algorithm, in preparation for talking about affine ciphers on Thursday.
Affine ciphers are a more complicated version of the shift cipher.
Then we'll have a computer programming makeup day: finish as much
programming from previous days as possible that wasn't finished. If
possible, start with factoring and prime checkers.
\item Suppose you're given the plaintext with first letter $p_{1}$, second
letter $p_{2}$, and so on. Remember that the Caesar or \emph{simple
shift} cipher takes each plaintext letter and adds some constant amount
$k$ to it. 
\item The \textbf{affine cipher} first \emph{multiplies }the plaintext letter
by a certain amount $m$, then adds a constant amount $k$. 
\item \includegraphics{\string"Old Cryptography Notes/pasted43\string".png}
\item In order to understand the affine cipher, we need to learn some \textbf{modular
arithmetic}. You've secretly been doing this already every time you've
encrypted or decrypted a shift cipher.
\end{itemize}
\begin{example}
{[}slide{]} Use the tool at https://tinyurl.com/ModClock to perform
``clock arithmetic'', explaining that $12=0$ on clocks. For example:
\begin{itemize}
\item $12+1\equiv1\mod12$
\item Since $96=12\times8$, $2+96\equiv2\mod12$.
\item You can set different clock sizes! Ask students for a modulus and
to predict various computations on a clock with that modulus.
\end{itemize}
\end{example}

\begin{xca}
\textbf{(RQ2) }Set your clock to $5$ hours.\textbf{ }Please complete
the addition and multiplication tables modulo 5 on p105 of your Coursepack
by filling in each box with one of the numbers 0, 1, 2, 3, or 4. For
example, the $4+4$ box on the addition table is asking \textquotedbl on
a clock with the numbers 0-4 on it, what time is 4 hours after 4 o'clock?\textquotedbl{}
The 3x4 box on the multiplication table is asking \textquotedbl if
we start at 0 o'clock and go forward four hours three times, what
time will it be?\textquotedbl{}

Bring your best attempt at filling in the tables to class on the date
below for discussion. Remember, you won't be graded on correctness,
but if you get lost please write a few sentences describing what you
tried and why it didn't work.

\begin{tabular}{|c|c|c|c|c|c|}
\hline 
$\times$ & 0 & 1 & 2 & 3 & 4\tabularnewline
\hline 
\hline 
0 & 0 & 0 & 0 & 0 & 0\tabularnewline
\hline 
1 & 0 & 1 & 2 & 3 & 4\tabularnewline
\hline 
2 & 0 & 2 & 4 & 1 & 3\tabularnewline
\hline 
3 & 0 & 3 & 1 & 4 & 2\tabularnewline
\hline 
4 & 0 & 4 & 3 & 2 & 1\tabularnewline
\hline 
\end{tabular}
\end{xca}

\begin{itemize}
\item Let's start with a couple of definitions:
\end{itemize}
\begin{defn*}
~
\begin{enumerate}
\item For the rest of this class, we'll write ``$a\mid b$'', say \textbf{``a
divides b''}, if $b/a$ is an integer.
\item If we're working on a clock that goes from $1$ to $n$, we say we're
\textbf{working modulo $n$} and that $n$ is the \textbf{modulus}.
\item Integers $a$ and $b$ are \textbf{congruent modulo $\mathbf{n}$}
if $n\mid(a-b)$. (That is, if $(a-b)/n$ is an integer.) If two integers
are congruent modulo $n$, we write $a\equiv b$ mod $n$.
\item The process of writing an integer $a$ as an integer $b$ so that
$a\equiv b\mod n$ and $0\leq b\leq n-1$ is called \textbf{reducing
$a$ modulo $n$.}
\end{enumerate}
\end{defn*}
How does this relate to the clock activity we just did?
\begin{itemize}
\item What does it mean if $n\mid(a-b)$? This is equivalent to saying $\frac{a-b}{n}$
is a whole number, for example $4$. But if $\frac{a-b}{n}=4$, then
$a-b=4n$, so $a=b+4n$. But $n$ is our clock size, so we can add
and subtract $4n$ without changing what time it is. Hence $a$ and
$b$ represent the same time on an $n$-hour clock if and only if
$n\mid(a-b)$, so our definitions are equivalent!
\end{itemize}
\begin{example*}
~
\begin{enumerate}
\item $7\equiv7$ mod 21 since $21\mid(7-7)$ ($0/21=0$, an integer)
\item $14\equiv2$ mod 3 since $3\mid(14-2)$.
\item $2\equiv12$ mod 5 since $5\mid(2-12)$.
\end{enumerate}
\end{example*}
Usually, if we're thinking about integers modulo $n$, we only want
to worry about the integers $0,1,2,\dots,n-1$. For instance, $12\equiv2$
mod $5$, so we think about $2$ mod 5 instead of 12.

We have two main ways to reduce $a$ mod $n$:
\begin{enumerate}
\item If $a\geq0$, we can replace $a$ with its remainder when it's divided
by $n$. Continuing Example 1, $7\equiv7$ mod 21 since $7\div21=0$
with remainder 7 and $14\equiv2$ mod 13 since $14\div3=4$ with remainder
2.
\item We can add or subtract multiple copies of $n$ since $n\equiv0$ mod
$n$, which is especially helpful for $a<0$. For example, $-7\equiv3$
mod 5 since $-7+2(5)=3$.
\end{enumerate}
The following table shows integers $x$ reduced modulo $5$ to $0,1,2,3,$
and $4$:

\includegraphics{\string"Old Cryptography Notes/pasted26\string".png}
\begin{xca}
{[}slide{]} Use the letter-number chart at the top of p30 of your
Coursepack to:
\begin{enumerate}
\item Reduce {[}i.e.. find the time on an $n$-hour clock{]} the following
numbers mod $n$:
\begin{enumerate}
\item $184\mod27$
\begin{enumerate}
\item We have that $184\div27=6R22$, so $184\equiv22\mod27$.
\end{enumerate}
\item $12\mod3$
\begin{enumerate}
\item Since $12/3=4R0$, $12\equiv0\mod3$.
\end{enumerate}
\item $0\mod17$
\begin{enumerate}
\item $0$ mod any positive number is $0$ because $0/\text{positive number}=0R0$!
\end{enumerate}
\item $3\mod1$.
\begin{enumerate}
\item Any whole number mod $1$ is $0$ because $x/1=xR0$ for any whole
number $x$!
\end{enumerate}
\end{enumerate}
\end{enumerate}
\end{xca}

\begin{example}
Now, say we want to find something like $1080+108\mod108$. One way
of interpreting this is ``if we start at $0:00$, go forward $1080$
hours, then go forward $108$ hours, what time is it on a $108$-hour
clock?
\begin{itemize}
\item One way of solving this problem is by adding $1080+108=1188$, then
reducing $1188\mod108$.
\item Can anyone think of an easier way?
\begin{itemize}
\item Notice that going forward $1080$ hours brings us back to $0:00$,
since $1080\equiv0\mod108$.
\item Notice also that going forward $108$ hours on a $108$-hour clock
doesn't change the time. So we're still at $0$:$00$!
\item Thus, $1080+108\equiv0\mod108$.
\end{itemize}
\item We've just discovered the following important fact about modular arithmetic:
\begin{itemize}
\item if we're trying to reduce $a+b\mod n$, we can first reduce $a$,
then $b$, then add the results.
\end{itemize}
\end{itemize}
\end{example}

\begin{xca}
What about multiplication? That is, can we say that
\[
1080(108)\equiv0(0)=0\mod108?
\]
Think about the clock interpretation of $1080\times108$ on a $108$-hour
clock. What about of time are we going forward by, and how many times
are we going forward this amount? Then reduce the following:
\begin{enumerate}
\item $90+15\mod15$

Since $90=15\times6$, we have
\[
90+15\equiv15\times6+15\equiv0\times6+0=0\mod15.
\]

\item $90\times15\mod15$

Similarly, we have~$90\times15\equiv0\times0=0\mod15$.
\item $1155\times444\mod2$

Since odd numbers are congruent to $1$ and even numbers are congruent
to $0$ mod $2$, we have
\[
1155\times444\equiv1\times0=0\mod2.
\]
In particular, this says that the result of $1155\times444$ is even,
which may not be surprising.
\end{enumerate}
\end{xca}

\begin{itemize}
\item Now, we note that modular arithmetic gives us a method of encrypting/decrypting
messages using shift ciphers without having to count backwards/forwards
in the alphabet:
\end{itemize}
\begin{xca}
{[}slide/handout{]} Encrypt ``gogriffins'' using a shift of $-24$
by doing the following:
\begin{enumerate}
\item Write out the plaintext message in the first row of a table, with
a different column for each letter.
\begin{enumerate}
\item Use the letter-number chart at the top of p30 of your Coursepack to
convert each letter of the plaintext to a number. Write the numbers
under the corresponding letter.
\item Can you convert the backwards shift of $24$ to a forwards shift that's
easier to perform? Write the corresponding forward shift under each
number.
\item How does (c) relate to the process of reducing $-24\mod26$?
\item Add the plaintext numbers and your forward shifts together and write
the resulting ciphertext numbers in the row below. {[}illustrate the
table{]}
\item Use the letter-number chart to convert each ciphertext number to a
ciphertext letter.
\end{enumerate}
\begin{tabular}{|c|c|c|c|c|c|c|c|c|c|c|}
\hline 
plaintext letter & g & o & g & r & i & f & f & i & n & s\tabularnewline
\hline 
\hline 
plaintext number & 6 & 14 & 6 & 17 & 8 & 5 & 5 & 8 & 13 & 18\tabularnewline
\hline 
$C\equiv P-24\equiv P+2\mod26$ & 8 & 16 & 8 & 19 & 10 & 7 & 7 & 10 & 15 & 20\tabularnewline
\hline 
ciphertext letter & I & Q & I & T & K & H & H & K & P & U\tabularnewline
\hline 
\end{tabular}
\item Use the same strategy to decrypt ``JDMCQHBJKZLZQADRSQZOODQZKHUD''
given that it was encrypted with a shift cipher with shift $+25$.

Our encryption equation is $C\equiv P+25\equiv P-1\mod26$. So our
decryption equation is $P\equiv C+1\mod26$. Shifting every letter
forward by $1$ yields the plaintext ''kendricklamarbestrapperalive''.
\item Finish the Arithme-Tic-Toc worksheet (Coursepack p105-109)
\end{enumerate}
\end{xca}

~
\begin{xca}
{[}slide{]} Work on shift ciphers Python modules
\end{xca}

~
\begin{xca}
Worksheet: Affine Cipher 1 (Coursepack p110)
\begin{enumerate}
\item (TPS) Assign student groups to create the multiplication and addition
tables for moduli other than 5 and 12. Students can investigate their
assigned tables for mathematical ideas and present their findings
to the class. In whole-class discussions, ask students to describe
how they discovered their patterns. Encourage and validate a variety
of appropriate responses. Using the activity sheet as a guide, ask
groups to write down the following: 
\begin{itemize}
\item A summary of what they found, as well as why they think their findings
are accuhadrate. 
\begin{itemize}
\item An explanation of any patterns they found. 
\end{itemize}
\end{itemize}
\end{enumerate}
\end{xca}

\begin{xca*}
1.1(a-c), 7(a-c), 8(a-c), 10 {[}Challenge problem!{]}, 15a, 16a
\begin{xca*}
\includegraphics{\string"Old Cryptography Notes/pasted136\string".png}
\end{xca*}
\end{xca*}

\end{document}
