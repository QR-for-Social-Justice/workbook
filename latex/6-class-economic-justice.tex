%% LyX 2.3.6.2 created this file.  For more info, see http://www.lyx.org/.
%% Do not edit unless you really know what you are doing.
\documentclass[oneside,english]{amsart}
\usepackage[T1]{fontenc}
\usepackage{amsthm}
\usepackage{graphicx}

\makeatletter
%%%%%%%%%%%%%%%%%%%%%%%%%%%%%% Textclass specific LaTeX commands.
\numberwithin{equation}{section}
\numberwithin{figure}{section}
\theoremstyle{plain}
\newtheorem{thm}{\protect\theoremname}
\theoremstyle{definition}
\newtheorem{xca}[thm]{\protect\exercisename}

\makeatother

\usepackage{babel}
\providecommand{\exercisename}{Exercise}
\providecommand{\theoremname}{Theorem}

\begin{document}
\title{WCSBS 220 5 - Class \& Economic Justice}
\maketitle
\begin{xca}
(RQ) To prepare for the next part of the \textquotedbl Who Makes
the Minimum Wage?\textquotedbl{} worksheet, you should familiarize
yourself with the Federal and state minimum wage laws. Please visit
the following three pages of the Department of Labor website and answer
the questions below. 
\begin{itemize}
\item Questions and Answers About the Minimum Wage 
\item Minimum Wage Laws in the States 
\item History of Changes to the Minimum Wage Law
\end{itemize}
\begin{enumerate}
\item Why are there Federal minimum wage and state minimum wages, and which
is used in states where the two are different? 
\item Which states have the highest and lowest minimum wages? 
\item Who is exempt from minimum wage requirements? 
\item What else have you learned about minimum wage? 
\item What else would you like to learn about minimum wage?
\end{enumerate}
\end{xca}

\begin{itemize}
\item Economic justice has been defined as \textquotedblleft a set of moral
principles for building economic institutions, the ultimate goal of
which is to create an opportunity for each person to create a sufficient
material foundation upon which to have a dignified, productive, and
creative life beyond economics.\textquotedblright{} (weirdly, the
source cited by Boston University's School of Public Health on this
is the investment tips site Investopedia.)
\item ``Therefore, an economic justice argument focuses on the need to
ensure that everyone has access to the material resources that create
opportunities, in order to live a life unencumbered by pressing economic
concerns. 
\begin{itemize}
\item Definitionally, this recalls the broader view of health expressed
by the World Health Organization: \textquotedblleft A state of complete
physical, mental, and social well-being and not merely the absence
of disease or infirmity.\textquotedblright{} 
\item In both cases, the pursuit of health and economic justice aspires
to something greater than simply physical well-being or financial
solvency.
\item The goal is, rather, to shape the fundamental conditions\textemdash i.e.
higher incomes, or freedom from preventable disease\textemdash that
allow people to live fulfilling, sustainable lives free from concerns
about meeting basic needs, or about falling into poor health.'' \textendash BU
\end{itemize}
\end{itemize}
\begin{xca}
What connections between economic justice and the COVID-19 pandemic
do you notice and wonder about? How does this expanded definition
of ''health'' agree and disagree with your personal definition?
\end{xca}

\begin{itemize}
\item {[}slide{]} the eroding of the American dream
\item ``There are several current efforts animating the public debate that
could be seen as approaches to achieve economic justice, and that
could, with such a focus, rise to the top of our agenda. Perhaps the
most direct of these is the universal basic income (UBI), a regular,
guaranteed payment made to each citizen, regardless of employment
or economic status.
\item ``The concept of economic justice intersects with the idea of overall
economic prosperity. There is a belief that creating more opportunities
for all members of society to earn viable wages will contribute to
sustained economic growth. When more citizens are able to provide
for themselves and maintain stable discretionary income, they are
more likely to spend their earnings on goods, which in turn drives
demand in the economy.'' \textendash Investopedia
\end{itemize}
\begin{xca}
Do you agree or disagree with Investopedia's argument above, and why?
How might their own positionality affect the framework through which
they view economic justice? How might greater economic justice increase
the standard of living for all? 

What are some policies you've heard, organizations, and/or protests
that you've been to that advocate for economic justice?
\end{xca}

\begin{itemize}
\item The opposite of economic justice is \textbf{economic injustice}: the
lack of opportunities for ''each person to create a sufficient material
foundation upon which to have a dignified, productive, and creative
life beyond economics'' {[}Investopedia{]}.
\item {[}slides{]} the mathematics of economic injustice
\end{itemize}
\begin{xca}
\textbf{Who Makes the Minimum Wage? }worksheet Part 1
\begin{enumerate}
\item Assume that a family is supported by a single minimum wage earner
who works 40 hours a week. Is the above claim true? 
\item How many hours would a minimum-wage earner have to work in a month
to be able to afford a two-bedroom apartment at FMR? Use a graph of
pay vs. time to illustrate the answer the question. 
\item How many hours a week, and how many weeks in a year would the person
from the previous question have to work? There are multiple answers
to this question. 4. How many hours would a minimum-wage earner have
to work in a month to be able to afford a two bedroom apartment at
FMR if s/he also worked overtime? Overtime pay, for working over 40
hours a week, is 1.5 times the regular pay. Use a graph of pay vs.
time again to illustrate the answer. 5. What would the minimum wage
have to be in order for a minimum wage earner working 40 hours a week
to be able to afford a 2-bedroom apartment? Can you answer this question
using a graph? If yes, please solve it in this way. If not, explain
why not. 6. In 29 states and Washington, D.C., the state minimum wage
is above the Federal minimum wage of \$7.25/hour. In Washington State,
for example, the minimum wage is \$12/hour. Does this mean that housing
is more affordable in other states with higher minimum wages? Explain. 
\item They want to work enough ($x$ hours/month) in order for their income
($(\$7.25)(12x)$ \$/month) to be sufficiently high that $30\%$ of
their income ($0.3(\$7.25)(12x)$) is at least \$1176:
\begin{align*}
0.3(\$7.25)(12x) & \geq\$1176\\
x & \geq\frac{\$1176}{0.3(\$7.25)}
\end{align*}
\end{enumerate}
\end{xca}

~
\begin{xca}
Who Makes the Minimum Wage? Part 2
\begin{enumerate}
\item . Which of the three numbers above is most helpful to you in understanding
the numbers of people who make minimum wage or less? Explain. 
\begin{enumerate}
\item Probably the ''number of workers at or below minimum wage total''
column
\end{enumerate}
\item In Table 1 on the linked page, look at the entire \textquotedblleft percent
distribution at or below minimum wage total\textquotedblright{} column.
Which numbers in this column add up to 100 and why? Which numbers
add up to each other and why? 
\begin{enumerate}
\item Don't add up to 100: Race/ethnicity, because some people are multiracial.
Add up to 100: Age categories.

\includegraphics[scale=0.4]{pasted8}
\end{enumerate}
\item ~
\begin{enumerate}
\item In the \textquotedblleft age and gender\textquotedblright{} portion
of the table, which numbers in the \textquotedblleft 25 years and
over\textquotedblright{} row add up to each other and why? 
\begin{enumerate}
\item Under Number of Workers (in thousands), ''at minimum wage'' + ''below
minimum wage'' = total.
\end{enumerate}
\item What other numbers in this row are related to each other through addition,
subtraction, multiplication, or division? Why does that relationship
hold? 
\item In the \textquotedblleft age and gender\textquotedblright{} portion
of the table, how are the \textquotedblleft percent distribution\textquotedblright{}
and the \textquotedblleft percent of workers paid hourly rates\textquotedblright{}
columns different? 
\end{enumerate}
\item Scroll down the table to the \textquotedblleft RACE AND HISPANIC OR
LATINO ETHNICITY\textquotedblright{} portion. Add up the numbers in
the \textquotedblleft percent distribution at or below minimum wage
total\textquotedblright{} column for White, Black or African American,
Asian, and Hispanic or Latino. Do these numbers add up to 100\%? Why
or why not? 
\begin{enumerate}
\item No, because Latino is an ethnic category that people choose in addition
to a race on the American Community Survey, which is the source for
this data
\end{enumerate}
\item ~
\begin{enumerate}
\item Which category in the \textquotedblleft RACE AND HISPANIC OR LATINO
ETHNICITY\textquotedblright{} portion of the table has the highest
percentage of those earning minimum wage or less? whi
\begin{enumerate}
\item White, with $72.6\%$ of those earning at or below the minimum wage.
But this is expected due to the higher population of whites in the
US. A more relevant question: what \textbf{percentage }of whites earning
hourly wages earn at or below the minimum wage? This calculation would
be
\[
\frac{1164}{62461}=1.86\%
\]
compared to
\[
\frac{287}{12063}=2.37\%
\]
of Black folks.
\end{enumerate}
\item (b) Which category has the lowest percentage? 
\item (c) What is the difference between these two percentages? 
\item (d) Is this difference substantial? Justify your answer. (Hint: You
may want to look at the ratio between the two numbers, and not just
their difference.) 
\item (e) Is this gap surprising? Why or why not? What are its real world
implications? 
\end{enumerate}
\item ~
\begin{enumerate}
\item In the \textquotedblleft FULL- AND PART-TIME STATUS\textquotedbl{}
portion of the table, approximately what fraction of minimum wage
workers work full time? 
\item What fraction work part time? 
\item What is the ratio between the two? 
\item What can you conclude from this fraction and ratio? 
\end{enumerate}
\end{enumerate}
\end{xca}


\end{document}
